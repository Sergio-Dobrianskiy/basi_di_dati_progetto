La seguente descrizione riporta in linguaggio naturale i requisiti per il nostro sistema informativo:

\begin{adjustwidth}{\indent1}{\indent1}
"La software house CityNet© richiede un sistema informativo per la gestione del servizio CityCard che consenta agli utenti di acquistare abbonamenti e accedere a servizi ed eventi forniti da enti affiliati. CityCard può essere emessa sia in formato fisico che virtuale (online) e ha una durata di 5 anni, l'emissione della carta è gratuita e ogni utente ne può avere attiva al massimo una.
L'utilizzo della CityCard necessita la sottoscrizione di un abbonamento a scelta tra tre tipologie, ciascuna con una durata e un prezzo specifici. 
L'abbonamento permette agli utenti di acquistare servizi o prenotare eventi forniti dagli enti partner con uno sconto che varia dalla durata dell'abbonamento.

Il servizio deve essere fornito su un sito al quale l'utente si deve registrare.

Gli utenti si suddividono in tre categorie: cliente, fornitore e admin o amministratore. Una persona può creare più di un account.

Gli enti sono gestiti da utenti di tipo "fornitore", i quali, una volta associati a un ente, hanno la possibilità di creare nuovi servizi o eventi. 

Al primo login gli utenti di tipo "cliente" possono solamente ottenere una CityCard e gestire le carte di credito. Dopo l'attivazione potranno accedere anche alla sezione abbonamenti. Una volta sottoscritto l'abbonamento potranno accedere a tutti i servizi.

I clienti potranno aggiungere più di una carta di credito e renderne una predefinita per i pagamenti.
I clienti possono acquistare i servizi forniti utilizzando la carta di credito predefinita e godendo dello sconto già nominato in precedenza. La CityCard permette anche di partecipare agli eventi gratuiti e di usufruire gratuitamente del trasporto pubblico. 
I clienti, inoltre, avranno la possibilità di lasciare una recensione sui servizi acquistati.

L'utente di tipo "admin" ha la facoltà di bannare gli altri utenti, resettare le recensioni di un ente e visualizzare statistiche globali.

L'utente di tipo "fornitore" deve poter creare enti nuovi, associarsi ad uno di essi, creare servizi ed eventi e visualizzare statistiche dell'ente con il quale è associato.

Tutti gli utenti devono poter aggiornare le informazioni del proprio account.

Gli eventi devono essere singoli o periodici. 

Si deve tenere traccia dei check-in falliti.

Il sistema dovrà essere progettato per garantire una gestione efficace delle diverse funzionalità descritte, assicurando un'esperienza utente intuitiva e fluida per tutte le tipologie di utenti."
\end{adjustwidth}
\begin{itemize}
    \item 
    \textbf{Ban} (Lega ADMIN e USER)\\
    Un utente di tipo admin ha la facoltà di interdire l'accesso al sito
    ad un altro utente. L'associazione è di tipo 0-N
    con la cardinalità minima a 0 in quanto un admin può non bannare nessuno.
    \item 
    \textbf{Creazione ente} (lega FORNITORE e ENTE)\\
    Un utente di tipo fornitore ha la possibilità di creare un Ente a cui associarsi.
    L'associazione è di 0-N in quanto può anche non creare un Ente ma l'Ente può essere creato da un solo utente fornitore (associazione 1-1).
    \item 
    \textbf{Lavoro} (lega FORNITORE e ENTE)\\
    Come nell'associazione precedente, la cardinalità però diventa 0-1 in quando un utente di tipo fornitore può lavorare solo per un Ente.
    \item 
    \textbf{Organizzazione} (lega ENTE e EVENTO)\\
    \item 
    \textbf{Partecipazione} (lega CLIENTE e EVENTO)\\
    L'associazione è di tipo 0-N in quanto un cliente può non partecipare a nessun evento come partecipare a molti.
    \item 
    \textbf{Possiede carta di credito} (lega  CLIENTE e CARTA DI CREDITO)\\
    Un cliente può registrare sul portale più carte (associazione 0-N)
    \item 
    \textbf{Possiede CityCard} (lega CLIENTE e CITYCARD)\\
    Un utente può richiedere più CityCard ma può esserne attiva solo una alla volta
    \item 
    \textbf{Transazione} (lega CARTA DI CREDITO e CITYCARD e SERVIZIO)\\
    L'associazione transazione collega tre entità, la transazione può far riferimento sia all'entità CityCard che a quella di Servizio (permette di poter acquistare)
    \item 
    \textbf{Sottoscrive} (lega ABBONAMENTO e CITYCARD e CARTA DI CREDITO) \\
    Stessa triangolazione di prima.
    \item 
    \textbf{Riduzione} (lega ABBONAMENTO e SCONTO)
    L'associazione è di tipo 0-N in quanto lo stesso sconto si può applicare su più abbonamenti così come a nessuno.
\end{itemize}
L’obiettivo del progetto è di realizzare una piattaforma di registrazione ad un servizio di acquisto di una carta elettronica, la CityCard.
CityCard è un sistema che favorisce l’interazione tra turisti e fornitori di servizi. I tipi di servizi possono essere diversi.

Per accedere al sito web i clienti dovranno creare un account fornendo le loro generalità.
Una volta registrato l’utente base  potrà comprare una CityCard che avrà una validità limitata. Ogni CityCard avrà associato un cliente, una carta di credito e un saldo che potrà essere usato per comprare e attivare servizi. 

Di default la CityCard permette l’utilizzo del servizio di trasporto pubblico urbano. Inoltre potrà venire utilizzata per comprare servizi forniti dai fornitori di vario genere come visite guidate ai musei, biglietti utilizzabili sui trasporti extra urbani. 

All’interno del portale web il cliente potrà vedere in homepage una lista di eventi organizzati dai fornitori. 
Gli eventi potranno essere dei semplici annunci di servizio oppure dei periodi di tempo nei quali potranno essere acquistabili servizi di una certa categoria con uno sconto.
Potrà anche visualizzare la lista di fornitori con i relativi servizi.

Si prevede l'esistenza di tre tipi di account: cliente, fornitore, admin.

L'utente potrà vedere la parte pubblica del servizio, consultare i servizi disponibili, avrà la possibilità di lasciare una recensione per il servizio usufruito.

Gli account dei fornitori potranno invece rendere disponibili servizi, visualizzare statistiche relative ai propri servizi tra cui gli acquisti.

Gli account degli amministratori avranno accesso completo al database e potranno gestire gli utenti e accedere a statistiche. 

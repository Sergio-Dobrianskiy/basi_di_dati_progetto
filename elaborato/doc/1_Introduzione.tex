L’obiettivo del progetto è di realizzare un servizio web di acquisto e utilizzo di una carta elettronica chiamato CityCard.
CityCard è un sistema che favorisce l’interazione tra turisti e fornitori. Con fornitore si intende chiunque organizzi eventi o fornisca servizi.
I servizi e gli eventi proposti saranno a discrezione dei fornitori, ma potranno spaziare dalla vendita di visite guidate a spettacoli teatrali. 
Servizi ed eventi si distinguono in quanto i primi saranno ad occorrenza singola e avranno un costo mentre gli eventi saranno gratuiti e avranno la possibilità di essere periodici.
Inoltre la CityCard permette fornisce l'accesso gratuito a tutti i mezzi di trasporto pubblico della zona interessata.

Per accedere al sito web i clienti dovranno creare un account fornendo le loro generalità. Una volta registrato l’utente base potrà attivare una CityCard che avrà una validità limitata. 
Ogni CityCard avrà associato un cliente e una carta di credito predefinita con e dovrà venire attivata sottoscrivendo un abbonamento. 
L'abbonamento, oltre a permettere partecipare ad eventi, comprare servizi e l'utilizzo dei trasporti pubblici, fornirà uno sconto sul prezzo dei servizi. 


Si prevede l'esistenza di tre tipi di account: cliente, fornitore, admin.

All’interno del portale web il cliente potrà vedere una lista degli eventi, dei servizi disponibili ed effettuare il check-in sui trasporti pubblici. 
L'utente, inoltre, avrà la possibilità di lasciare una recensione per il servizio acquistati.

I fornitori potranno invece creare enti, associarsi ad uno di essi, rendere disponibili servizi ed eventi, e visualizzare statistiche relative all'ente al quale sono associati.

Gli account degli amministratori potranno visualizzare gli utenti registrati e in caso bannarli, visualizzare gli enti registrati e resettare le loro recensioni, e consultare statistiche relative al servizio.
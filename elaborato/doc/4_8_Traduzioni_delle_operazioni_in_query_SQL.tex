In questa sezione tradurremo tutte le singole operazioni elementari che abbiamo finora analizzato solo
attraverso gli schemi di navigazione dell’E/R in vere e proprie query in linguaggio SQL. 
Da notare che per facilitare lo dell'applicazione web alcune query sono multiple e di conseguenza è stata posta l'opzione multipleStatements.

\\
\subsubsection{Recuperare tutti gli utenti}
\begin{lstlisting}
    SELECT id_user, nome, cognome, email, bannato, descrizione_ruolo as ruolo
    FROM users u
    JOIN ruoli r 
    on u.id_ruolo = r.id_ruolo;
\end{lstlisting}

\subsubsection{Recuperare tutti gli enti con e senza recensioni}
seleziona tutti gli enti con e senza recensione, COALESCE restituisce 0 se non c'è una media, i due JOIN imitano una FULL JOIN .
\begin{lstlisting}
    SELECT e.id_ente, e.nome as nome_ente, descrizione, e.indirizzo as indirizzo_ente, numero_telefono, e.id_user, u.nome as nome_user, cognome, email, cf, telefono, id_ruolo, media_recensioni
    FROM enti e
    JOIN users u
    on e.id_user = u.id_user
    left JOIN (
        SELECT  s.id_ente as id_ente,  avg(coalesce(votazione, 0)) as media_recensioni
        FROM recensioni r
        JOIN servizi s
        on s.id_servizio = r.id_servizio
        GROUP BY s.id_ente
    ) as r
    on r.id_ente = e.id_ente 
    union
    SELECT e.id_ente, e.nome as nome_ente, descrizione, e.indirizzo as indirizzo_ente, numero_telefono, e.id_user, u.nome as nome_user, cognome, email, cf, telefono, id_ruolo, media_recensioni
    FROM enti e
    JOIN users u
    on e.id_user = u.id_user
    right JOIN (
        SELECT  s.id_ente as id_ente,  avg(coalesce(votazione, 0)) as media_recensioni
        FROM recensioni r
        JOIN servizi s
        on s.id_servizio = r.id_servizio
        GROUP BY s.id_ente
    ) as r
    on r.id_ente = e.id_ente
\end{lstlisting}


\subsubsection{Recupero lista eventi}
\begin{lstlisting}
    SELECT ev.*, en.nome as organizzatore, periodi.*
    FROM eventi ev
    JOIN enti en
    on ev.id_ente = en.id_ente
    JOIN periodi
    on periodi.id_periodo = ev.id_periodo
    WHERE now() < ev.fine_validita
    ORDER BY ev.inizio_validita DESC;
\end{lstlisting}

\subsubsection{Recupero lista servizi}
\begin{lstlisting}
    set @id_user = ?;     
    set @cityCard = ( SELECT id_city_card
            FROM city_card
            WHERE id_user = @id_user and data_scadenza > now());
    set @percentToPay = (SELECT (100 - sconti.percentuale_sconto) / 100
                FROM sconti 
                JOIN listino_abbonamenti ls 
                on sconti.id_sconto = ls.id_sconto
                JOIN sottoscrizioni_abbonamento sa
                on  ls.id_sconto = sa.id_listino_abbonamento
                WHERE sa.id_city_card = @cityCard);
    SELECT se.*, en.nome as organizzatore, se.prezzo_servizio * @percentToPay as prezzo_scontato
            FROM servizi se
            JOIN enti en
            on se.id_ente = en.id_ente
            WHERE now() < se.fine_validita;
\end{lstlisting}

\subsubsection{Recupero acquisti di un utente}
\begin{lstlisting}
    SELECT sa.data_acquisto, sa.prezzo_pagato, cc.id_city_card, sa.num_carta_credito, servizi.descrizione_servizio as nome_servizio, u.id_user, sa.id_servizio
    FROM servizi_acquistati sa
    JOIN city_card cc
    on cc.id_city_card = sa.id_city_card
    JOIN users u
    on u.id_user = cc.id_user
    JOIN servizi
    on servizi.id_servizio = sa.id_servizio
    WHERE cc.data_scadenza>now() and cc.id_user = ?
    ORDER BY sa.data_acquisto DESC;
\end{lstlisting}



\subsubsection{Reitorna statistiche amministratore}
Uso le variabili per rendere più leggibili le query e le unisco in un unico select per rendere l'inserimento in tabello più semplice. 
\begin{lstlisting}
    INSERT INTO users (username, nome, cognome, email,
    password, id_ruolo, data_creazione) 
    VALUES (SET @numeroCheckin = (SELECT count(coalesce(id_check,0)) FROM checks);
    SET @numeroCheckinFalliti = (SELECT count(coalesce(id_check,0)) FROM checks WHERE id_check != 1);
    SET @numeroCityCardAttive = (SELECT count(coalesce(id_city_card,0)) FROM city_card WHERE data_scadenza > now());
    SET @numeroEventiAttivi = (SELECT count(coalesce(id_evento,0)) FROM eventi WHERE fine_validita > now());
    SET @numeroServiziAttivi = (SELECT count(coalesce(id_servizio,0)) FROM servizi WHERE fine_validita > now());

    SELECT 
        @numeroCheckin as numero_checkin,
        @numeroCheckinFalliti as numeroCheckinFalliti,
        @numeroCityCardAttive as numeroCityCardAttive,
        @numeroEventiAttivi as numeroEventiAttivi,
        @numeroServiziAttivi as numeroServiziAttivi);
\end{lstlisting}





\subsubsection{Recupero statistiche fornitore}
\begin{lstlisting}
    set @id_utente = 2;
    set @id_ente = ( SELECT c.id_ente
    FROM collaborazioni c
    WHERE c.id_user = @id_utente and c.fine_collaborazione is null limit 1);
    
    SET @saldo = (SELECT saldo FROM enti WHERE id_ente = @id_ente);
    SET @numeroEventiAttivi = (SELECT count(coalesce(id_evento,0)) FROM eventi WHERE fine_validita > now() and id_ente = @id_ente);
    SET @numeroServiziAttivi = (SELECT count(coalesce(id_servizio,0)) FROM servizi WHERE fine_validita > now() and id_ente = @id_ente);


    SELECT  
        ifnull(@saldo,0) as saldo,
        @numeroEventiAttivi as numeroEventiAttivi,
        @numeroServiziAttivi as numeroServiziAttivi;
\end{lstlisting}




\subsubsection{Recupero listino abbonamenti}
\begin{lstlisting}
    SELECT l.*, s.percentuale_sconto
    FROM listino_abbonamenti l
    JOIN sconti s
    on l.id_sconto = s.id_sconto
    WHERE l.data_disattivazione is null or l.data_disattivazione < now();
\end{lstlisting}


\subsubsection{Creazione nuovo ente}
\begin{lstlisting}
    INSERT INTO enti (nome, descrizione, indirizzo, numero_telefono, id_user) 
    VALUES (?, ?, ?, ?,  ?);
\end{lstlisting}









\subsubsection{Inserimento voto utente}
Dato che un utente può avere una recensione per servizio cancello la inserisco il voto nuovo, calcolo la media, la inserisco nel servizio.
\begin{lstlisting}
    SET @idUser = ?;
    SET @idServizio = ?;
    
    DELETE 
    FROM recensioni 
    WHERE id_user = @idUser and id_servizio = @idServizio;
    
    insert into recensioni (id_user, votazione, id_servizio) VALUES(@idUser,?,@idServizio);
    
    SET @media = (SELECT AVG(votazione) 
        FROM recensioni 
        WHERE id_user = @idUser and id_servizio = @idServizio);
    
    UPDATE servizi
    SET media_recensioni = @media
    WHERE id_servizio = @idServizio;
\end{lstlisting}



\subsubsection{Reset recensioni di un ente}
Reset di tutte le recensioni riguardanti un ente
\begin{lstlisting}
    SET @id_ente = ?;

    DELETE recensioni 
    FROM recensioni
    JOIN servizi s
    on s.id_servizio = recensioni.id_servizio
    WHERE s.id_ente = @id_ente;

    UPDATE servizi s
    SET media_recensioni = null
    WHERE s.id_ente = @id_ente;
\end{lstlisting}




\subsubsection{Creazione servizio}
Crea un servizio, "LIMIT 1" non dovrebbe servire in quanto un fornitore può essere
associato con solo un ente ma protegge da possibili errori come una doppia scrittura dello stesso record
\begin{lstlisting}
    INSERT INTO servizi (descrizione_servizio, indirizzo_servizio, id_ente, prezzo_servizio) 
    VALUES (?, ?, 
        ( SELECT c.id_ente
        FROM collaborazioni c
        WHERE c.id_user = ? and c.fine_collaborazione is null LIMIT 1), ?);
\end{lstlisting}

\subsubsection{Registrazione nuovo utente}
\begin{lstlisting}
    INSERT INTO users (username, nome, cognome, email, password, id_ruolo, data_creazione) 
    VALUES (?,?,?,?,?,?,?);
\end{lstlisting}

\subsubsection{Registrazione nuova carta di credito}
\begin{lstlisting}
    INSERT INTO carte_credito (num_carta_credito, cognome_associato, nome_associato, mese_scadenza, anno_scadenza, id_user) 
    VALUES (?,?,?,?,?,?);
\end{lstlisting}

\subsubsection{Creazione nuova CityCard}
\begin{lstlisting}
    INSERT INTO city_card (id_user) 
    VALUES('?');
\end{lstlisting}

\subsubsection{Disattivazione CityCard}
Per disattivare una card pongo la data di scadenza a now().
Per controllare se la card è attiva controllo se il momento di disattivazione è già passato.
In ogni momento un utente può avere al massimo una CityCard attiva,
quindi per default disattivo tutte quelle attive.
\begin{lstlisting}
    update city_card
    set data_scadenza = now()
    WHERE id_user = ? and data_scadenza > now();
\end{lstlisting}

\subsubsection{Rendere predefinita una carta di credito}
rendo predefinita una carta di credito di un utente. Rendo non predefinite tutte le altre. Query doppia, una delle alternative a quello fatto in ban
\begin{lstlisting}
    update carte_credito
    set predefinita = 0
    WHERE id_user = ?;

    update carte_credito
    set predefinita = 1
    WHERE num_carta_credito = ?;
\end{lstlisting}

\subsubsection{Sottoscrizione abbonamento}
\begin{lstlisting}
    INSERT INTO sottoscrizioni_abbonamento (id_listino_abbonamento, id_city_card, num_carta_credito) 
    VALUES (?, 
            ( SELECT id_city_card
                        FROM city_card
                        WHERE id_user = ? and data_scadenza > now()), 
            ( SELECT num_carta_credito 
                        FROM carte_credito
                        WHERE id_user = ? and predefinita = 1));
\end{lstlisting}

\subsubsection{Associa utente a ente}
\begin{lstlisting}
    INSERT INTO collaborazioni 
    (id_user, id_ente) 
    VALUES (?, ?);`
\end{lstlisting}

\subsubsection{Effettua check-in}
\begin{lstlisting}
    INSERT INTO checks (id_city_card, id_mezzo, id_stato) 
    VALUES (( SELECT id_city_card
                FROM city_card
                WHERE id_user = ? and data_scadenza > now()), 
                ?, ?);`
\end{lstlisting}

\subsubsection{Ottieni CityCard di un utente}
\begin{lstlisting}
    SELECT *, (if (data_scadenza > now(), "attiva", "non attiva")) as stato
    FROM city_card 
    WHERE id_user = ?
    ORDER BY id_city_card DESC
\end{lstlisting}

\subsubsection{Partecipa a un evento}
\begin{lstlisting}
    INSERT INTO partecipazioni (id_evento, id_user) VALUES (?,?);
\end{lstlisting}

\subsubsection{Edita Utente}
\begin{lstlisting}
    UPDATE users 
    SET username = ?, nome = ?, cognome = ?, email = ?, password = ?, indirizzo = ?, telefono = ?, cf = ? 
    WHERE id_user = ?;
\end{lstlisting}

\subsubsection{Banna/sbanna utente}
Query tecnicamente sbagliata ma ho provato ad usare il trucco dell'INNER JOIN per aggirare il problema. L'alternativa è fare due query separate ban/unban.
\begin{lstlisting}
    UPDATE users u
    INNER JOIN users u1 
    on u.id_user = u1.id_user
    SET u.bannato = CASE
    WHEN (
       (
       SELECT u1.bannato
       WHERE u1.id_user = ?
       ) = 1
        ) THEN 0
     ELSE 1
     END
    WHERE u.id_user = ?;
\end{lstlisting}

\subsubsection{Cancella carta di credito}
\begin{lstlisting}
    DELETE FROM carte_credito
    WHERE num_carta_credito = ?;
\end{lstlisting}

\subsubsection{Effettua login}
\begin{lstlisting}
    SELECT u.*, r.descrizione_ruolo as ruolo
    FROM users u
    JOIN ruoli r 
    on u.id_ruolo = r.id_ruolo
    WHERE username = ? AND password = ?;
\end{lstlisting}

\subsubsection{Ritorna, se presente, la CityCard attiva dell'utente}
\begin{lstlisting}
    SELECT *
    FROM city_card
    WHERE id_user = ? and data_scadenza > now();
\end{lstlisting}

\subsubsection{Ritorna ente associato all'utente}
\begin{lstlisting}
    SELECT *
    FROM collaborazioni
    WHERE id_user = ? and fine_collaborazione is null;
\end{lstlisting}


\subsubsection{ritorna, se presente, la sottoscrizione attiva attiva dell'utente}
\begin{lstlisting}
    SELECT sa.* 
    FROM sottoscrizioni_abbonamento sa
    JOIN city_card cc
    on sa.id_city_card = cc.id_city_card
    JOIN users u
    on cc.id_user = u.id_user
    WHERE u.id_user = ?;
\end{lstlisting}


\subsubsection{Ritorna lista check-in fatti da un utente}
\begin{lstlisting}
    SELECT ch.*, sc.descrizione_stato, m.*
    FROM checks ch
    JOIN city_card cc
    on ch.id_city_card = cc.id_city_card
    JOIN users u
    on cc.id_user = u.id_user
    JOIN stati_check sc
    on ch.id_stato = sc.id_stato
    JOIN mezzi m
    on ch.id_mezzo = m.id_mezzo
    WHERE u.id_user = ?;`
\end{lstlisting}


\subsubsection{Recupera enti}
\begin{lstlisting}
    SELECT * FROM enti
    WHERE nome like ?;
\end{lstlisting}




\subsubsection{Recupera carte di credito di un utente}
\begin{lstlisting}
    SELECT c.*, u.nome, u.cognome
    FROM carte_credito c
    JOIN users u
    on c.id_user = u.id_user
    WHERE c.id_user = ?;
\end{lstlisting}


\subsubsection{Compra servizio}
Compra un servizio tenendo conto dello sconto concesso dalla fascia di abbonamento posseduto, poi aggiunge la somma pagata al saldo dell'ente che eroga il servizio.
La query è stata fatta con le variabili perché senza incappava nello stesso problema del ban utente, cioè conflitto in caso di stesso attributi in INSERT E WHERE della stessa query.
\begin{lstlisting}
    set @id_user = ?;
    set @id_servizio = ?;
    set @idEnte = (
                        SELECT enti.id_ente 
                        FROM enti
                        JOIN servizi
                        on enti.id_ente = servizi.id_ente
                        WHERE servizi.id_servizio = @id_servizio);
    set @creditCard = ( SELECT num_carta_credito 
                        FROM carte_credito
                        WHERE id_user = @id_user and predefinita = 1);
    set @cityCard = (   SELECT id_city_card
                        FROM city_card
                        WHERE id_user = @id_user and data_scadenza > now());
    set @percentToPay = (SELECT (100 - sconti.percentuale_sconto) / 100
                        FROM sconti 
                        JOIN listino_abbonamenti ls 
                        on sconti.id_sconto = ls.id_sconto
                        JOIN sottoscrizioni_abbonamento sa
                        on  ls.id_sconto = sa.id_listino_abbonamento
                        WHERE sa.id_city_card = @cityCard);
    set @paidPrice = (  SELECT prezzo_servizio * @percentToPay
                        FROM servizi
                        WHERE id_servizio = @id_servizio);

    INSERT INTO servizi_acquistati (id_servizio, prezzo_pagato, num_carta_credito, id_city_card) 
    VALUES (@id_servizio, @paidPrice, @creditCard,@cityCard );
    
    UPDATE enti 
    set saldo = saldo + @paidPrice
    WHERE id_ente = @idEnte;
\end{lstlisting}




\subsubsection{Creazione evento periodico}
Per ottenere l'ultimo ID inserito in periodi una opzione poteva essere l'utilizzo di LAST{\_}INSERT{\_}ID ma questo richiedeva l'attivazione di PersistentConnections al momento della connessione.

\begin{lstlisting}
    set @idUser = ?;
    set @idEnte = ( SELECT c.id_ente
            FROM collaborazioni c
            WHERE c.id_user = @idUser and c.fine_collaborazione is null limit 1);

    INSERT INTO periodi (lunedi, martedi, mercoledi, giovedi, venerdi, sabato, domenica) 
    VALUES (?,?,?,?,?,?,?);
    
    set @idPeriodo = (SELECT MAX(id_periodo) FROM periodi);

    INSERT INTO eventi (id_periodo, nome_evento, num_partecipanti, id_ente) 
    VALUES (@idPeriodo, ?,?, @idEnte);
\end{lstlisting}


\subsubsection{Creazione evento occasionale}
\begin{lstlisting}
    set @idUser = ?;
    set @idEnte = ( SELECT c.id_ente
            FROM collaborazioni c
            WHERE c.id_user = @idUser and c.fine_collaborazione is null limit 1);

    INSERT INTO eventi (id_periodo, nome_evento, num_partecipanti, id_ente) 
    VALUES (1, ?,?, @idEnte);
\end{lstlisting}
\subsubsection{Entità e Attributi}
Lo schema va analizzato per individuarne le parole e le espressioni chiave, con queste verrà realizzato un primo schema riassuntivo che verrà raffinato in seguito. I termini di rilievo appaiono nel testo con un colore diverso:

\begin{adjustwidth}{\indent1}{\indent1}
"La software house CityNet© richiede un sistema informativo per la gestione del servizio \speciale{CityCard} che consenta agli utenti di acquistare \speciale{abbonamenti} e accedere a \speciale{servizi} ed \speciale{eventi} forniti da enti affiliati. CityCard può essere emessa sia in formato fisico che virtuale (online) e ha una durata di 5 anni, l'emissione della carta è gratuita e ogni utente ne può avere attiva al massimo una.

Gli abbonamenti disponibili con CityCard si suddividono in tre \speciale{tipologie}, ciascuna con una durata e un prezzo specifici. L'abbonamento permette agli utenti di acquistare servizi o prenotare eventi forniti dagli \speciale{enti} partner con uno sconto che varia dalla durata dell'abbonamento.

Il servizio deve essere fornito su un \speciale{sito} al quale l'\speciale{utente} si deve registrare.

Gli utenti si suddividono in tre categorie: \speciale{cliente}, \speciale{fornitore} e \speciale{admin} o amministratore. Una persona può creare più di un account.

Gli enti sono gestiti da utenti di tipo "fornitore", i quali, una volta associati a un ente, hanno la possibilità di creare nuovi servizi o eventi. 

Al primo login gli utenti di tipo "cliente" possono solamente ottenere una CityCard e gestire le carte di credito. Dopo l'attivazione potranno accedere anche alla sezione abbonamenti. Una volta sottoscritto all'abbonamento potranno accedere a tutti i servizi.

I clienti potranno aggiungere più di una \speciale{carta di credito} e renderne una predefinita per i pagamenti.

I clienti possono acquistare i servizi forniti utilizzando la carta di credito predefinita e godendo dello \speciale{sconto} già nominato in precedenza. Inoltre la CityCard permette di partecipare agli eventi gratuiti e di usufruire gratuitamente del trasporto pubblico. I clienti, inoltre, avranno la possibilità di lasciare una \speciale{recensione} sul servizio acquistato o sull'evento a cui ha partecipato.

L'utente di tipo "admin" ha la facoltà di \speciale{bannare} gli altri utenti. I clienti, invece, possono acquistare i servizi offerti e prenotare eventi attraverso il sistema.

Il sistema dovrà essere progettato per garantire una gestione efficace delle diverse funzionalità descritte, assicurando un'esperienza utente intuitiva e fluida per tutte le tipologie di utenti."
\end{adjustwidth}
\medskip
La descrizione del prodotto è già abbastanza chiara, ma presenta alcune entità che vengono nominate più di una volta usando dei sinonimi, sarà quindi importante chiarire che si sta parlando sempre della stessa entità e non confondersi. 



%TODO TABELLA SINONIMI%
%TODO Rivedere larghezza colonne%
%IDEA: segnare in un altro stile attributi%
\begingroup % localize the following settings      
\setlength{\arrayrulewidth}{0.5mm}
\renewcommand{\arraystretch}{1.5}
 \rowcolors{2}{PaleTurquoise}{white}
\begin{longtblr}
[
  caption = {Estrazione delle entità principali},
  label = {tab:Estrazione delle entità principali},
]{
  colspec = {|XXX[2]|},
  rowhead = 2,
  hlines,
  row{even} = {PaleTurquoise},
  row{1} = {SkyBlue},
} 
Termine & Sinonimi usati & Descrizione\\
CityCard & Carta servizi & x Tessera che permette l'acquisto di un abbonamento\\
Clienti & Visitatori & Persone che usufruiscono di servizi offerti da un ente\\
Fornitori & Venditore, rifornitore & Persone che, per conto di terzi, fornisce servizi per la clientela \\
Admin & Amministratore & Persona responsabile della gestione del sito \\
Servizi & Attività, prestazioni & Operazioni svolte per soddisfare le esigenze dei clienti \\
Abbonamento & Sottoscrizione & Contratto che prevede un pagamento una tantum per poter accedere ai servizi\\
Ente & Associato & Un'azienda o un'entità con accordi per offrire servizi \\
Eventi & Incontri & Attività organizzate per riunire persone in occasione specifiche \\
Recensioni & Valutazioni & Commenti o giudizi espressi dai clienti su servizi \\
Sconto & Riduzione & Ribasso del prezzo originale di un prodotto o servizio.\\
\end{longtblr}


Lo schema concettuale nella sua versione finale si avvarrà delle seguenti entità e associazioni:




\begin{longtblr}
[
  caption = {Entità e associazioni},
  label = {tab:Entità e associazioni},
]{
  colspec = {|XXX[2]|},
  rowhead = 2,
  hlines,
  row{even} = {PalePink},
  row{1} = {pink},
} 
Nome & Tipo & Descrizione\\
Utente & E & Possessore di un account \\
Servizio & E & Attività messa a disposizione da un fornitore \\
Ente & E & Associazioni che forniscono servizi (es. Musei) \\
Evento & E & Avvenimento gratuito con o senza prenotazione \\
Cliente & E & Colui che potrà usufruire di servizi \\
Fornitore & E & Erogatore servizi o eventi \\
Transazione & A & Acquisto di un servizio \\
Recensione & E & Valutazione di un servizio, evento o fornitore \\
Carta di credito & E & Mezzo con il quale si effettua la transazione \\
Possiede City Card & A & Lega l'utente alla city card a lui assegnata \\
Possiede Carta Credito & A & Lega l'utente alla sua carta di credito \\
Abbonamento & E & Permette l'utilizzo del trasporto pubblico \\
Organizza & A & Attività che lega il fornitore all'utente garantendo servizi o eventi \\
Lavora & A & Indica per quale/i ente/i collabora il fornitore  \\
Admin & E & Utente che ha il controllo della piattaforma \\
Partecipa & A & I clienti partecipano agli eventi \\
Sottoscrive & A & Attività di un utente che si "abbona" ad uno o più fornitori per seguire le novità \\
% TODO: Aggiungere allo schema ABBONA
Abbona & A & Un cliente si abbona ad un abbonamento \\
CityCard & E & Carta che permette di usufruire dei servizi \\
% TODO: Aggiungere allo schema COMITIVA
% TODO: Aggiungere allo schema RAGGRUPPA
Percentuale sconto  & A & L'esatto importo di riduzione di prezzo\\
Sconti & E & Riduzione di prezzo di un servizio \\
Prenotazione & A & Attività che lega un utente e uno o più eventi laddove è necessaria confermare la propria presenza\\
*Trasporti pubblici*  & E & Lista dei trasporti da interrogare \\
Viaggia & A &  Lista utilizzi CityCard 
\end{longtblr}


%%%% TABELLA ATTRIBUTI%%%
%%%% TABELLA ATTRIBUTI%%%

% \setlength{\arrayrulewidth}{0.5mm}
% \renewcommand{\arraystretch}{1.5}
%  \rowcolors{2}{PaleGreen}{white}

\endgroup
%TODO: decidere e in caso aggiungere una descrizione estesa di tutte le entità e attributi
\subsubsection{Analisi dettagliata di entità e attributi}

\textbf{Utente:} L'utente sarà l'utilizzatore generico del servizio e al database al quale sarà appoggiato. 
Ogni utente dovrà appartenere ad una delle seguenti categorie. 
\begin{itemize}
    \item Cliente: può visionare e comprare i servizi offerti
    \item Fornitore: può creare e offrire servizi
    \item Admin: ha accesso totale al database e può compiere azioni necessarie per la moderazione e la manutenzione del sistema.
\end{itemize}
Le informazioni minime associate a ciascun utente saranno: 
\begin{itemize}
    \item Nome
    \item Cognome
    \item Codice fiscale
    \item email
\end{itemize}

\textbf{Ente:} L'ente sarà l'azienda partner che metterà a disposizione i suoi servizi e gli eventi organizzati in favore dei clienti.
Le informazioni minime associate a ciascun ente saranno:
\begin{itemize}
    \item Nome
    \item Descrizione dell'ente
    \item La data di creazione
\end{itemize}

\textbf{Servizio:} Il servizio è la macro area che il fornitore mette a disposizione nella piattaforma.
Un servizio avrà queste caratteristiche.
\begin{itemize}
    \item Fornitore: sarà l'ente che creerà il servizio
\end{itemize}

\textbf{Evento:} Con evento si intende l'attività specifica all'interno del servizio. Ad esempio, si può acquistare
\begin{itemize}
    \item Prezzo: il costo per poterne usufruire
    \item Data di validità: data in cui potrà usufruire dell'evento o il range di date disponibili
\end{itemize}

\textbf{Abbonamento:} L'abbonamento è una sottoscrizione tramite la quale il cliente potrà accedere alla lista eventi e a quella dei servizi per poterli prenotare o acquistare. Ci sono tre tipi di abbonamento tra cui scegliere, tutte con una durata e un prezzo specifico.
Un abbonamento avrà le seguenti caratteristiche:
\begin{itemize}
    \item Descrizione del tipo di abbonamento
    \item Prezzo
    \item Periodo di validità
\end{itemize}

\textbf{CityCard:} La CityCard è una tessera fisica o virtuale che permette la sottoscrizione di un abbonamento.
Un utente può avere solo una CityCard alla volta ed avrà la durata massima di 5 anni.
Una CityCard avrà le seguenti caratteristiche:
\begin{itemize}
    \item Data di emissione
    \item Data di scadenza
    \item Numero identificativo della tessera
\end{itemize}

\textbf{Ban di un utente:} La possibilità di bannare un utente è prerogativa dell'admin che, a sua discrezione, può intercedere un utente dall'utilizzo della piattaforma. Se l'utente viene bannato non gli è più possibile fare il login.
Caratteristiche di questa associazione:
\begin{itemize}
    \item Essere un admin
    \item Il cliente e il fornitore hanno un attributo booleano "bannato" il cui valore cambia in base allo stato in cui si trova (attivo o disattivato).
\end{itemize}

\textbf{Carta di credito:}La carta di credito è lo strumento con la quale si può acquistare un abbonamento, ogni cliente può avere più carte di credito associate al suo account ma solo una può essere predefinita per gli acquisti.
Le caratteristiche della carta di credito sono le seguenti:
\begin{itemize}
    \item Numero della carta di credito
    \item Mese e anno di scadenza
    \item La data di scadenza
    \item Un attributo che chiarisce quale carta di credito è quella di default
\end{itemize}


%TODO: mettiamo solo le principali o tutte, ad esempio serve login per tutti gli account o calcoliamo solamente per l'utente normale


\vspace{0.5cm}
\subsubsection{Funzionalità richieste dall'applicativo}
Di seguito un elenco delle funzionalità richieste per l'\ul{cliente}:
\begin{enumerate}[label=c\arabic*)]
    \item Richiedere una CityCard
    \item Sottoscrivere un abbonamento per una CityCard
    \item Aggiungere carte di credito al proprio account
    \item Rendere predefinita una carta di credito
    \item Acquistare un servizio
    \item Prenotare un evento
    \item Lasciare rating ai servizi acquistati (1-5 stelle)
    \item Effettuare checkin per il mezzo di trasporto scelto
\end{enumerate}

Di seguito un elenco delle funzionalità richieste per un \ul{amministratore}:
\begin{enumerate}[label=a\arabic*)]
    \item Visualizzare lista utenti
    \item Bannare tutti i tipi di account
    \item Cancellare enti
    \item Resettare recensioni di un ente
    \item Consultare le statistiche
\end{enumerate}

\vspace{0.5cm}

%TODO: mettiamo solo le principali o tutte, ad esempio serve login per tutti gli account o calcoliamo solamente per l'utente normale
Di seguito un elenco delle funzionalità richieste per un \ul{fornitore}: 
\begin{enumerate}[label=b\arabic*)]
    \item Creare enti
    \item Associasi a un ente
    \item Creare servizi
    \item Creare eventi
    \item Consultare statistiche riguardo l'ente al quale è associato 
    \item Consultare il saldo
    
\end{enumerate}


%Lo schema dell'applicativo potrà risultare un po' vasto da qui la nostra scelta di illustrare le varie entità volta per volta.

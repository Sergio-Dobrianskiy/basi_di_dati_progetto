In questa fase ci occuperemo dell'analisi delle ridondanze cioè quella informazioni che possono essere ricavate altrove.
Una ridondanza quindi corrisponde ad un dato che può essere derivato, cioè ottenuto attraverso una serie di operazioni, da altri dati.
Se si mantiene la ridondanza si semplificano alcune interrogazioni ma si occupa maggior spazio.
All'interno del nostro schema possiamo fare alcune considerazioni.

\begin{itemize}
    \item L'attributo numero{\_}partecipanti è derivabile da una operazione di conteggio delle istanze di clienti che hanno prenotato un evento.
    Dobbiamo tenere conto delle frequenze di esecuzione.
    \begin{itemize}
        \item operazione 1: si tiene il conteggio ogni volta che un cliente prenota un evento
        \item operazione 2: vengono visualizzati tutti i dati di un evento
    \end{itemize}
\end{itemize}
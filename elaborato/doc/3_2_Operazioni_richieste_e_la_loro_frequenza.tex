In questa fase esponiamo una tabella delle operazioni utilizzate per costituire una stima delle principali operazioni richieste dal sito. 



\begingroup % localize the following settings  
\setlength{\arrayrulewidth}{0.5mm}
\renewcommand{\arraystretch}{1.5}


\begin{longtblr}
[
  caption = {Operazioni richieste da tutti gli User},
  label = {tab:Operazioni richieste da tutti gli User},
]{
  colspec = {|X[1]X[8,l]X[3]X[1]X[8,l]|},
  rowhead = 1,
  hlines,
  row{even} = {lightgray},
  row{1} = {ColdPurple},
} 
Cod & Nome Operazione & Freq & Tipo & Descrizione\\
u.1 & Registrarsi alla piattaforma & \num{300} & I & Registrazione al sito\\ 
u.2 & Accedere tramite login & \num{50} & I & Login dell'account \\ 
u.3 & Modificare i dati del proprio account & \num{50} & I & Login dell'account 
\end{longtblr}



\begin{longtblr}
  [
    caption = {Operazioni richieste amministratore},
    label = {tab:Operazioni richieste amministratore},
  ]{
    colspec = {|X[1]X[8,l]X[3]X[1]X[8,l]|},
    rowhead = 1,
    hlines,
    row{even} = {lightgray},
    row{1} = {CornflowerBlue},
  } 
  Cod & Nome Operazione & Freq & Tipo & Descrizione\\
  a.1 & Consultare la lista degli enti & \num{50} & B & Può consultare la lista degli enti registrati \\
  a.2 & Bannare gli altri utenti & \num{50} & I & L'admin può interdire l'accesso alla piattaforma \\ 
  a.3 & Reset delle recensioni & \num{30} & I & Può cancellare le recensioni \\
  a.4 & Consultazione statistiche & \num{30} & I & Può consultare statistiche globali della piattaforma \\
  \end{longtblr}
  

  \begin{longtblr}
    [
      caption = {Operazioni richieste da Fornitore},
      label = {tab:Operazioni richieste da Fornitore},
    ]{
      colspec = {|X[1]X[8,l]X[3]X[1]X[8,l]|},
      rowhead = 1,
      hlines,
      row{even} = {lightgray},
      row{1} = {LightCoral},
    } 
    Cod & Nome Operazione & Freq & Tipo & Descrizione\\
    f.1 & Creare enti & \num{5} & I & Il fornitore crea enti \\ 
    f.2 & Associarsi a un ente  & \num{5} & I & Il deve associarsi al suo ente \\ 
    f.3 & Creare servizi & \num{5}/giorno & I & Il fornitore va a creare servizi\\
    f.4 & Creare eventi occasionali & \num{5} & I & Il fornitore crea eventi \\ 
    f.5 & Creare eventi periodici & \num{5} & I & Il fornitore crea eventi \\ 
    f.6 & Consultare statistiche riguardo il proprio ente  & \num{50} & B & Il fornitore va a consultare una tabella contenente i dati\\ 
    
    \end{longtblr}


\begin{longtblr}
[
  caption = {Operazioni richieste dai Clienti},
  label = {tab:Operazioni richieste da cliente},
]{
  colspec = {|X[1]X[8,l]X[3]X[1]X[8,l]|},
  rowhead = 1,
  hlines,
  row{even} = {lightgray},
  row{1} = {ColdPurple},
} 
Cod & Nome Operazione & Freq & Tipo & Descrizione\\
c.1 & Richiedere una CityCard & \num{100} & I & Il cliente può richiedere una CityCard \\ 
c.2 & Sottoscrivere un abbonamento & \num{100} & I & Il cliente sottoscrive un abbonamento per utilizzare la piattaforma \\ 
c.3 & Aggiungere una carta di credito & \num{100} & I & Il cliente memorizza una carta di credito nella piattaforma \\ 
c.4 & Rendere una carta di credito predefinita & \num{100} & I & Il cliente rende predefinita la carta che utilizzera per gli acquisti \\ 
c.5 & Acquistare un servizio & \num{100} & I & Il cliente può acquistare un servizio \\ 
c.6 & Prenotare un evento & \num{100} & I & Chiede all'applicativo una disponibilità di eventi \\
c.7 & Effettuare un check-in & \num{100} & I & Chiede all'applicativo una disponibilità di eventi \\ 
c.8 & Consultare la lista degli acquisti fatti & \num{80} & B & Consulta una lista \\ 
c.9 & Lasciare una recensione riguardo un servizio acquistato & \num{70} & I & Il cliente rilascia una votazione \\ 

\end{longtblr}
% I = Interattiva
% B = Batch



\endgroup
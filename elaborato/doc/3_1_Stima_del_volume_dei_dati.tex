Si fornisce in questa fase una tabella contenente il numero medio di istanze di ogni entità e associazione dello schema globale.

\begingroup % localize the following settings  
\setlength{\arrayrulewidth}{0.5mm}
\renewcommand{\arraystretch}{1.5}
\begin{longtblr}
[
  caption = {Stima del volume di dati},
  label = {tab:Stima del volume di dati},
]{
  colspec = {|X[1.5]XXX[4]|},
  rowhead = 2,
  hlines,
  row{even} = {lightgray},
  row{1} = {ColdPurple},
} 
Concetto & Costrutto & Volume & Descrizione\\
Utente & Entità & \num{10000} & Si stima che i clienti, gli enti e gli amministratori siano numerosi \\
Possiede CityCard & Relazione & \num{9000} & I clienti potrebbero registrarsi ma non avere una CityCard attiva \\
Ente & Entità & \num{500} & Fornisce i servizi che il cliente usufruisce\\
Crea ente & Relazione & \num{500} & Ogni ente è creato da un fornitore \\
Lavora & Relazione & \num{400} & Non tutti i fornitori lavorano per un ente \\
Servizi & Entità & \num{1500} & Data la cardinalità mi aspetto una stima di 3 servizi ad ente \\
Abbonamento & Entità & \num{9000} & La stima è fatta su cardinalità minima 1, ogni cliente possiede un abbonamento\\
Sottoscrizione & Relazione & \num{9000} & La cardinalità minima è 1-1 quindi per ogni CityCard è presente un abbonamento \\
Carta di credito & Entità & \num{10000} & Si stima che per ogni cliente sia associata una carta di credito\\
Evento & Entità & \num{1000} & Si stima che vengano organizzati almeno 2 eventi per ente \\
Organizzazione evento & Relazione & \num{1000} & Essendo cardinalità 1-1 per ogni evento è prevista un'organizzazione\\
Partecipazione & Relazione & \num{900} & Non tutti i clienti possono essere interessati ad uno specifico evento\\
Recensione & Relazione & \num{1200} & Ogni cliente può lasciare una recensione per servizio o evento ma non è detto che tutti i clienti lascino un'opinione \\
Trasporto pubblico & Entità & \num{2} & Sono presenti due tipi di trasporto previsti all'interno dell'abbonamento
\end{longtblr}
\endgroup
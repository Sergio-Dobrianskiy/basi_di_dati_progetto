\subsubsection{Installazione su windows}
\begin{itemize}
\item Scaricare e installare Node.js: https://nodejs.org/en/download/prebuilt-installer
\item spostarsi in /basi{\_}di{\_}dati{\_}progetto/webapp/server
\item 'node install'
\item npm install -g nodemon
\end{itemize}


\subsubsection{restore database}
Dalla cartella 'web{\_}app' lanciare il comando:
\begin{itemize}
\item su windows: 'Get-Content dump.sql | mysql -uroot -p1234 web{\_}app'
\item su unix: 'mysql -uroot -p1234 \texttt{<} dump.sql'
\end{itemize}

\subsubsection{Creazione utente database}
Il restore dovrebbe creare l'utente necessario per far funzionare l'applicativo, se così non fosse: creare un utente chiamato 'web{\_}app' password '1234' e concedere tutti i diritti.

\subsubsection{avvio app}
\begin{itemize}
\item  posizionarsi in /basi{\_}di{\_}dati{\_}progetto/webapp/
\item  'npm start' oppure 'nodemon src/app'
\item  andare a http://localhost:5000/
\end{itemize}

\subsubsection{Utenti applicazione}
Durante l'utilizzo è possibile creare gli utenti per utilizzare l'applicazione, altrimenti ci sono già presenti tre utenti:
\begin{itemize}
    \item  username: admin \hspace{1.5cm} password: admin
    \item  username: fornitore \hspace{1.5cm} password: fornitore
    \item  username: cliente \hspace{1.5cm} password: cliente
\end{itemize}
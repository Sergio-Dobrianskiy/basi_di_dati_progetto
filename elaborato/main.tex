\documentclass{article}
\usepackage{graphicx} % Required for inserting images
\usepackage[dvipsnames]{xcolor}
\usepackage{changepage} % required for adjustwidth
\usepackage{soul} % required for \ul, underline doesn't wrap
\usepackage{color, colortbl}
\usepackage{longtable}
\usepackage{tabu}
\usepackage{tabularray}
\usepackage{siunitx}  % required for \num 
\usepackage{enumitem}% http://ctan.org/pkg/enumitem
\usepackage{fmtcount}
\usepackage{pifont,amssymb} % for the symbols
\usepackage{listings} % parti di codice sql
\usepackage{ulem} % dasuline

\lstset{ % listings config
  % basicstyle=\ttfamily,
  % emphstyle=\textbf
  columns=fullflexible,
  frame=single,
  breaklines=true,
  postbreak=\mbox{\textcolor{red}{$\hookrightarrow$}\space},
}
% \usepackage[shortlabels]{enumitem}
\graphicspath{ {./images/} }
% le prossime 4 righe rendono doc cartella default per input
% \makeatletter
% \providecommand*{\input@path}{}
% \edef\input@path{{./doc/}\input@path}% prepend
% \makeatother

\def\indent1{0.3cm}
\newcommand{\speciale}[1]{\textbf{\textcolor{blue}{\ul{#1}}}}
\newcommand{\specialeb}[1]{\textit{\textcolor{ForestGreen}{\dashuline{#1}}}}



\definecolor{Gray}{gray}{0.9}
\definecolor{PaleGreen}{RGB}{152,251,152}
\definecolor{SkyBlue}{RGB}{135,206,235}
\definecolor{PaleTurquoise}{RGB}{179,238,238}
\definecolor{MediumAquamarine}{RGB}{102,205,170}
\definecolor{ColdPurple}{RGB}{171, 160, 217}
\definecolor{LightCoral}{RGB}{240, 128, 128}
\definecolor{CornflowerBlue}{RGB}{100, 149, 237}
\definecolor{MediumSeaGreen}{RGB}{60, 179, 113} % cliente
\definecolor{PalePink}{RGB}{253, 215, 244}

\newlist{answerlist}{enumerate}{2}
\setlist[answerlist]{label={\alph*.\makebox[0pt][r]{\noexpand\emptysquare\hspace{2em}}},ref=\alph*}

\newcommand{\emptysquare}{$\square$}
\newcommand{\checkedsquare}{\makebox[0pt][l]{\raisebox{1pt}[0pt][0pt]{\large\hspace{1pt}\cmark}}$\square$}
\newcommand{\cmark}{\ding{51}}%
\newcommand{\correctanswer}{{\renewcommand{\emptysquare}{\checkedsquare}\item\leavevmode}}


% https://erd.dbdesigner.net/designer/schema/1721817964-basi-di-dati

%%%%%%%%%%%%%%%%%
% DOC BEGINS HERE
%%%%%%%%%%%%%%%%%

\title{Relazione Basi di Dati}
\author{
  Dobrianskiy, Sergio \\
  \texttt{sergio.dobrianskiy@studio.unibo.it}\\
  \texttt{0001019553}
  \and
  Rigato, Valentina\\
  \texttt{valentina.rigato@studio.unibo.it}\\
  \texttt{0001027266}
}
\date{01 Marzo 2024}

\begin{document} 
\maketitle
\includegraphics[width=0.95\columnwidth]{CesenaCard2.png}
\newpage

\tableofcontents
\clearpage

\section{Introduzione}
L’obiettivo del progetto è di realizzare una piattaforma di registrazione ad un servizio di acquisto di una carta elettronica, la CityCard.
CityCard è un sistema che favorisce l’interazione tra turisti e fornitori di servizi. I tipi di servizi possono essere diversi.

Per accedere al sito web i clienti dovranno creare un account fornendo le loro generalità.
Una volta registrato l’utente base  potrà comprare una CityCard che avrà una validità limitata. Ogni CityCard avrà associato un cliente, una carta di credito e un saldo che potrà essere usato per comprare e attivare servizi. 

Di default la CityCard permette l’utilizzo del servizio di trasporto pubblico urbano. Inoltre potrà venire utilizzata per comprare servizi forniti dai fornitori di vario genere come visite guidate ai musei, biglietti utilizzabili sui trasporti extra urbani. 

All’interno del portale web il cliente potrà vedere in homepage una lista di eventi organizzati dai fornitori. 
Gli eventi potranno essere dei semplici annunci di servizio oppure dei periodi di tempo nei quali potranno essere acquistabili servizi di una certa categoria con uno sconto.
Potrà anche visualizzare la lista di fornitori con i relativi servizi.

Si prevede l'esistenza di tre tipi di account: cliente, fornitore, admin.

L'utente potrà vedere la parte pubblica del servizio, consultare i servizi disponibili, avrà la possibilità di lasciare una recensione per il servizio usufruito.

Gli account dei fornitori potranno invece rendere disponibili servizi, visualizzare statistiche relative ai propri servizi tra cui gli acquisti.

Gli account degli amministratori avranno accesso completo al database e potranno gestire gli utenti e accedere a statistiche. 



\section{Analisi dei requisiti}

\subsection{Requisiti in linguaggio naturale}
La seguente descrizione riporta in linguaggio naturale i requisiti per il nostro sistema informativo:

\begin{adjustwidth}{\indent1}{\indent1}
"La software house CityNet© richiede un sistema informativo per la gestione del servizio CityCard che consenta agli utenti di acquistare abbonamenti e accedere a servizi ed eventi forniti da enti affiliati. CityCard può essere emessa sia in formato fisico che virtuale (online) e ha una durata di 5 anni, l'emissione della carta è gratuita e ogni utente ne può avere attiva al massimo una.
L'utilizzo della CityCard necessita la sottoscrizione di un abbonamento a scelta tra tre tipologie, ciascuna con una durata e un prezzo specifici. 
L'abbonamento permette agli utenti di acquistare servizi o prenotare eventi forniti dagli enti partner con uno sconto che varia dalla durata dell'abbonamento.

Il servizio deve essere fornito su un sito al quale l'utente si deve registrare.

Gli utenti si suddividono in tre categorie: cliente, fornitore e admin o amministratore. Una persona può creare più di un account.

Gli enti sono gestiti da utenti di tipo "fornitore", i quali, una volta associati a un ente, hanno la possibilità di creare nuovi servizi o eventi. 

Al primo login gli utenti di tipo "cliente" possono solamente ottenere una CityCard e gestire le carte di credito. Dopo l'attivazione potranno accedere anche alla sezione abbonamenti. Una volta sottoscritto l'abbonamento potranno accedere a tutti i servizi.

I clienti potranno aggiungere più di una carta di credito e renderne una predefinita per i pagamenti.
I clienti possono acquistare i servizi forniti utilizzando la carta di credito predefinita e godendo dello sconto già nominato in precedenza. La CityCard permette anche di partecipare agli eventi gratuiti e di usufruire gratuitamente del trasporto pubblico. 
I clienti, inoltre, avranno la possibilità di lasciare una recensione sui servizi acquistati.

L'utente di tipo "admin" ha la facoltà di bannare gli altri utenti, resettare le recensioni di un ente e visualizzare statistiche globali.

L'utente di tipo "fornitore" deve poter creare enti nuovi, associarsi ad uno di essi, creare servizi ed eventi e visualizzare statistiche dell'ente con il quale è associato.

Tutti gli utenti devono poter aggiornare le informazioni del proprio account.

Gli eventi devono essere singoli o periodici. 

Si deve tenere traccia dei check-in falliti.

Il sistema dovrà essere progettato per garantire una gestione efficace delle diverse funzionalità descritte, assicurando un'esperienza utente intuitiva e fluida per tutte le tipologie di utenti."
\end{adjustwidth}

\subsection{Estrazione dei concetti fondamentali}
\subsubsection{Entità e Attributi}
Lo schema va analizzato per individuarne le parole e le espressioni chiave, con queste verrà realizzato un primo schema riassuntivo che verrà raffinato in seguito. I termini di rilievo appaiono nel testo con un colore diverso:

\begin{adjustwidth}{\indent1}{\indent1}
"La software house CityNet© richiede un sistema informativo per la gestione del servizio \speciale{CityCard} che consenta agli \speciale{utenti} di \specialeb{acquistare} \speciale{abbonamenti} e \specialeb{accedere} a \speciale{servizi} ed \speciale{eventi} forniti da \speciale{enti} affiliati. \speciale{CityCard} può essere \specialeb{emessa} sia in formato fisico che virtuale (online) e ha una durata di 5 anni, l'\specialeb{emissione} della carta è gratuita e ogni \speciale{utente} ne può avere attiva al massimo una.

L'utilizzo della \speciale{CityCard} necessita la \specialeb{sottoscrizione} di un \speciale{abbonamento} a scelta tra tre \speciale{tipologie}, ciascuna con una durata e un prezzo specifici. L'\speciale{abbonamento} permette agli \speciale{utenti} di \specialeb{acquistare} \speciale{servizi} o \specialeb{prenotare} eventi forniti dagli \speciale{enti} partner con uno \speciale{sconto} che varia dalla \speciale{durata} dell'abbonamento.
Il servizio deve essere fornito su un \speciale{sito} al quale l'\speciale{utente} si deve \specialeb{registrare}.

Gli utenti si suddividono in tre categorie: \speciale{cliente}, \speciale{fornitore} e \speciale{admin} o \speciale{amministratore}. Una persona può \specialeb{creare} più di un \speciale{account}.

Gli \speciale{enti} sono gestiti da \speciale{utenti} di tipo "\speciale{fornitore}", i quali, una volta \specialeb{associati} a un \speciale{ente}, hanno la possibilità di \specialeb{creare} nuovi \speciale{servizi} o \speciale{eventi}. 

Al primo \specialeb{login} gli \speciale{utenti} di tipo "\speciale{cliente}" possono solamente \specialeb{ottenere} una \speciale{CityCard} e \specialeb{gestire} le \speciale{carte di credito}. Dopo l'\specialeb{attivazione} potranno accedere anche alla sezione \speciale{abbonamenti}. Una volta \specialeb{sottoscritto} l'\speciale{abbonamento} potranno accedere a tutti i \speciale{servizi}.

I \speciale{clienti} potranno aggiungere più di una \speciale{carta di credito} e renderne una predefinita per i pagamenti.

I \speciale{clienti} possono \specialeb{acquistare} i \speciale{servizi} forniti utilizzando la \speciale{carta di credito} predefinita e godendo dello \speciale{sconto} già nominato in precedenza. La \speciale{CityCard} permette anche di \specialeb{partecipare} agli \speciale{eventi} gratuiti e di \specialeb{usufruire} gratuitamente del \speciale{trasporto pubblico}. 
I clienti, inoltre, avranno la possibilità di \specialeb{lasciare} una \speciale{recensione} sui \speciale{servizi} \specialeb{acquistati}.

L'utente di tipo "\speciale{admin}" ha la facoltà di \specialeb{bannare} gli altri \speciale{utenti}, \specialeb{resettare} le \speciale{recensioni} di un \speciale{ente} e \specialeb{visualizzare} \speciale{statistiche} globali.

L'utente di tipo "\speciale{fornitore}" deve poter \specialeb{creare} enti \speciale{nuovi}, \specialeb{associarsi} ad uno di essi, \specialeb{creare} \speciale{servizi} ed \speciale{eventi} e \specialeb{visualizzare} \speciale{statistiche} dell'\speciale{ente} con il quale è associato.

Tutti gli \speciale{utenti} devono poter \specialeb{aggiornare} le \speciale{informazioni} del proprio \speciale{account}.

Gli \speciale{eventi} devono essere \speciale{singoli} o \speciale{periodici}. 

Si deve tenere traccia dei \speciale{check-in} falliti.

Il sistema dovrà essere progettato per garantire una gestione efficace delle diverse funzionalità descritte, assicurando un'esperienza utente intuitiva e fluida per tutte le tipologie di utenti."
\end{adjustwidth}
\medskip
La descrizione del prodotto è già abbastanza chiara, ma presenta alcune entità che vengono nominate più di una volta usando dei sinonimi, sarà quindi importante chiarire che si sta parlando sempre della stessa entità e non confondersi. 

\begingroup % localize the following settings      
\setlength{\arrayrulewidth}{0.5mm}
\renewcommand{\arraystretch}{1.5}
\rowcolors{2}{PaleTurquoise}{white}
\begin{longtblr}
[
    caption = {Estrazione delle entità principali},
    label = {tab:Estrazione delle entità principali},
]{
    colspec = {|XXX[2]|},
    rowhead = 2,
    hlines,
    row{even} = {PaleTurquoise},
    row{1} = {SkyBlue},
} 
Termine & Sinonimi usati & Descrizione\\
CityCard & & Tessera che permette l'acquisto di un abbonamento e la fruizione di servizi o eventi\\
Clienti & Utenti & Persone che usufruiscono di servizi offerti da un ente\\
Fornitori & & Persone associate a un ente che forniscono eventi e/o servizi \\
Admin & Amministratore & Persona responsabile della gestione del sito \\
Servizi & & Operazioni svolte per soddisfare le esigenze dei clienti \\
Abbonamento & Sottoscrizione & Contratto che prevede un pagamento una tantum per poter accedere ai servizi o eventi per un periodo\\
Ente & & Un'azienda o un'entità che fornisce eventi e servizi \\
Eventi & & Attività organizzate per riunire persone in occasione specifiche \\
Recensioni & & Commenti o giudizi espressi dai clienti su servizi \\
Sconto & & Ribasso del prezzo originale di un servizio.\\
Check-in & Usufruire & Convalida dell'utilizzo del trasporto pubblico.\\
\end{longtblr}


Lo schema concettuale nella sua versione finale si avvarrà delle seguenti entità e associazioni:




\begin{longtblr}
[
    caption = {Entità e associazioni},
    label = {tab:Entità e associazioni},
]{
    colspec = {|X[3,l]X[1]X[8]|},
    rowhead = 2,
    hlines,
    row{even} = {PalePink},
    row{1} = {pink},
} 
Nome & Tipo & Descrizione\\
Utente & E & Possessore di un account \\
Servizio & E & Attività a pagamento messa a disposizione da un fornitore \\
Evento & E & Avvenimento gratuito con o senza prenotazione \\
Ente & E & Associazioni che forniscono servizi (es. Musei) \\
Cliente & E & Colui che potrà usufruire di servizi ed eventi \\
Fornitore & E & Persona associata a un ente \\
Transazione & A & Acquisto di un servizio \\
Recensione & E & Valutazione di un servizio \\
Carta di credito & E & Mezzo con il quale si effettua la transazione \\
Possiede City Card & A & Lega l'utente alla city card a lui assegnata \\
Possiede Carta Credito & A & Lega l'utente alla sua carta di credito \\
Abbonamento & E & Permette l'utilizzo del trasporto pubblico \\
Abbona & A & Un cliente si abbona ad un abbonamento \\
Organizza & A & Attività che lega il fornitore all'utente garantendo servizi o eventi \\
Lavora & A & Indica per quale ente collabora il fornitore  \\
Admin & E & Utente che ha il controllo della piattaforma \\
Partecipa & A & I clienti partecipano agli eventi \\
Sottoscrive & A & Attività di un utente che si "abbona" ad uno o più fornitori per seguire le novità \\
Banna & A & L'amministratore banna un utente dal sito\\
CityCard & E & Carta che permette di usufruire dei servizi \\
Percentuale sconto  & A & L'esatto importo di riduzione di prezzo\\
Sconti & E & Riduzione di prezzo di un servizio \\
Prenotazione & A & Attività che lega un utente e uno o più eventi laddove è necessaria confermare la propria presenza\\
Trasporti pubblici & E & Lista dei trasporti da interrogare \\
Check-in & A & Lista utilizzi CityCard \\
\end{longtblr}
\endgroup

\subsubsection{Analisi dettagliata di entità e attributi}

\textbf{Utente:} L'utente sarà l'utilizzatore generico del servizio e al database al quale sarà appoggiato. 
Ogni utente dovrà appartenere ad una delle seguenti categorie. 
\begin{itemize}
    \item Cliente: può visionare e comprare i servizi offerti
    \item Fornitore: può creare e offrire servizi
    \item Admin: compiere azioni necessarie per la moderazione e la manutenzione del sistema.
\end{itemize}
Le informazioni minime associate a ciascun utente saranno: 
\begin{itemize}
    \item Nome
    \item Cognome
    \item Codice fiscale
    \item Username
    \item Password
    \item email
    \item stato: bannato o attivo
\end{itemize}

\textbf{Ente:} L'ente sarà l'azienda partner che metterà a disposizione i suoi servizi e gli eventi organizzati in favore dei clienti.
Le informazioni minime associate a ciascun ente saranno:
\begin{itemize}
    \item Nome
    \item Descrizione dell'ente
    \item La data di creazione
\end{itemize}

\textbf{Servizio:} Il servizio il prodotto che il fornitore mette a disposizione nella piattaforma.
Un servizio avrà queste caratteristiche.
\begin{itemize}
    \item Fornitore: sarà l'ente che creerà il servizio
    \item Descrizione
    \item Prezzo
    \item Attivo: il servizio può non essere più in vendita
\end{itemize}

\textbf{Evento:} L'evento è un'attività gratuita messa a disposizione da un ente.
Un ente avrà queste caratteristiche.
\begin{itemize}
    \item Fornitore: sarà l'ente che creerà il servizio
    \item Descrizione
    \item Data di validità: data in cui potrà usufruire dell'evento o il range di date disponibili
    \item Attivo: l'evento può non essere più disponibile
\end{itemize}

\textbf{Abbonamento:} L'abbonamento è una sottoscrizione tramite la quale il cliente potrà accedere alla lista eventi e a quella dei servizi per poterli prenotare o acquistare. Ci sono tre tipi di abbonamento tra cui scegliere, tutte con una durata e un prezzo specifico.
Un abbonamento avrà le seguenti caratteristiche:
\begin{itemize}
    \item Descrizione del tipo di abbonamento
    \item Prezzo
    \item Durata
\end{itemize}

\textbf{CityCard:} La CityCard è una tessera fisica o virtuale che permette la sottoscrizione di un abbonamento.
Un utente può avere solo una CityCard alla volta ed avrà la durata massima di 5 anni.
Una CityCard avrà le seguenti caratteristiche:
\begin{itemize}
    \item Data di emissione
    \item Data di scadenza
    \item Numero identificativo della tessera
\end{itemize}

\textbf{Ban di un utente:} La possibilità di bannare un utente è prerogativa dell'admin che, a sua discrezione, può intercedere un utente dall'utilizzo della piattaforma. Se l'utente viene bannato non gli è più possibile fare il login.
Caratteristiche di questa associazione:
\begin{itemize}
    \item Essere un admin
    \item Il cliente e il fornitore hanno un attributo booleano "bannato" il cui valore cambia in base allo stato in cui si trova (attivo o disattivato).
\end{itemize}

\textbf{Carta di credito:} La carta di credito è lo strumento con la quale si può acquistare un abbonamento, ogni cliente può avere più carte di credito associate al suo account ma solo una può essere predefinita per gli acquisti.
Le caratteristiche della carta di credito sono le seguenti:
\begin{itemize}
    \item Numero della carta di credito
    \item Mese e anno di scadenza
    \item Nome e Cognome del proprietario
    \item Un attributo che chiarisce quale carta di credito è quella di default
\end{itemize}




\vspace{0.5cm}
\subsubsection{Funzionalità richieste dall'applicativo}


Di seguito un elenco delle funzionalità richieste da tutti gli \ul{utenti}:
\begin{enumerate}[label=u\arabic*)]
    \item Registrarsi
    \item Effettuare un login
    \item Modificare i dati del proprio account
\end{enumerate}

\vspace{0.5cm}


Di seguito un elenco delle funzionalità richieste per un \ul{amministratore}:
\begin{enumerate}[label=a\arabic*)]
    \item Visualizzare lista utenti
    \item Bannare gli altri utenti
    \item Resettare recensioni di un ente
    \item Consultare le statistiche
\end{enumerate}

\vspace{0.5cm}


Di seguito un elenco delle funzionalità richieste per l'\ul{cliente}:
\begin{enumerate}[label=c\arabic*)]
    \item Richiedere una CityCard
    \item Sottoscrivere un abbonamento per una CityCard
    \item Aggiungere carte di credito al proprio account
    \item Rendere predefinita una carta di credito
    \item Acquistare un servizio
    \item Prenotare un evento
    \item Lasciare rating ai servizi acquistati (1-5 stelle)
    \item Effettuare checkin per il mezzo di trasporto scelto
\end{enumerate}



Di seguito un elenco delle funzionalità richieste per un \ul{fornitore}: 
\begin{enumerate}[label=b\arabic*)]
    \item Creare enti
    \item Associasi a un ente
    \item Creare servizi
    \item Creare eventi occasionali
    \item Creare eventi periodici
    \item Consultare statistiche riguardo l'ente al quale è associato 
\end{enumerate}

\subsection{Schema Iniziale}
Qui viene riportato uno schema provvisorio creato dopo una prima analisi dei requisiti e dei concetti fondamentali. Nei paragrafi successivi le entità e le relazioni verranno raffinate e lo schema verrà corretto.

\includegraphics[width=0.95\columnwidth]{images/Schema_Iniziale.png}

\subsection{Raffinamento dello schema ER}
\subsubsection{Modifica chiave primaria utente}
In seguito a una analisi approfondita dei requisiti, la necessità di permettere a una persona di poter creare più di un account e di poter modificare i propri dati rende comodo l'utilizzo di una chiave primaria numerica.  

\subsubsection{Scelta chiave primaria carta di credito}
L'univocità del numero della carta di credito ci permette di utilizzarlo come chiave primaria dell'entità.

\subsubsection{Scelta delle altre chiavi primarie delle altre tabelle}
Per semplicità è stato deciso di utilizzare una chiave primaria numerica in tutte le altre tabelle.

\subsection{Normalizzazione User} %TODO
Tolto attributo composto (primo livello)

\subsection{Normalizzazione evento} %TODO
Tolto visibile (secondo livello), si fa check fine validità

\subsection{Aggiunta relazione tra Abbonamento e Sconto} %TODO

\subsection{Aggiunta attributi minimi in Trasporti pubblici} %TODO

\subsection{Modifica tabella sconto} %TODO
Non è più uno storico di chi ha ricevuto lo sconto ma una lista di sconti disponibili.

\subsection{Aggiunta di CityCard alla relazione 'transazione'} %TODO
Citycard identifica l'utente che compra il servizio.

\subsection{Modifica tabella sconto} %TODO


\subsection{Schema Finale}
Qui viene riportato uno schema ER finale.

\begin{center}
\centerline{\includegraphics[width=0.9\paperwidth]{images/schema_ER_finale.png}}
\end{center}



\section{Progettazione Logica}

\subsection{Stima del volume dei dati}
Si fornisce in questa fase una tabella contenente il numero medio di istanze di ogni entità e associazione dello schema globale. \\

\begingroup % localize the following settings  
\setlength{\arrayrulewidth}{0.5mm}
\renewcommand{\arraystretch}{1.5}

\begin{longtblr}
[
  caption = {Stima del volume di dati},
  label = {tab:Stima del volume di dati},
]{
  colspec = {|X[3,l]X[2]X[2]X[6,l]|},
  rowhead = 1,
  hlines,
  row{even} = {PaleTurquoise},
    row{1} = {SkyBlue},
} 
Concetto & Costrutto & Volume & Descrizione\\
Utente & Entità & \num{100000} & Si stima che i clienti, gli enti e gli amministratori siano numerosi \\
Possiede CityCard & Relazione & \num{100000} & I clienti potrebbero registrarsi ma non avere una CityCard attiva e alcuni potrebbero avere più di una CityCard \\
Admin & Entità & \num{20} & Si stima che ci saranno 20 amministratori \\
Ente & Entità & \num{300} & Fornisce i servizi che il cliente usufruisce\\
Crea ente & Relazione & \num{300} & Ogni ente è creato da un fornitore \\
Fornitore & Entità & \num{800} & Si stima che ci siano 2-3 fornitori per ogni ente \\
Lavora & Relazione & \num{750} & Non tutti i fornitori lavorano per un ente \\
Servizi & Entità & \num{2400} & Data la cardinalità mi aspetto una stima di 8 servizi ad ente durante l'anno \\
Transazione & Relazione & \num{1000000} & Si stima che in media ogni utente compri 5 servizi \\
Abbonamento & Entità & \num{3} & In ogni momento ci saranno 3 tipi di abbonamento\\
Sottoscrizione & Relazione & \num{100000} & Si stima una media di un abbonamento per ogni utente \\
Carta di credito & Entità & \num{150000} & Si stima che per ogni cliente sia associata più una carta di credito\\
Evento & Entità & \num{1200} & Si stima che vengano organizzati almeno 4 eventi per ente \\
Organizzazione evento & Relazione & \num{1200} & Essendo cardinalità 1-1 per ogni evento è prevista un'organizzazione\\
Partecipazione & Relazione & \num{180000} & Non tutti i clienti possono essere interessati agli eventi \\
Recensione & Relazione & \num{500000} & Ogni cliente può lasciare una recensione per servizio acquistato ma non è detto che tutti i clienti lo facciano \\
Sconto & Entità &  \num{6} & In base al periodo di alta o bassa stagione ci saranno sconti diversi \\
Riduzione & Relazione &  \num{3} & In ogni momento ci sarà uno sconto per ogni abbonamento\\
Ban & Relazione &  \num{100} & Si stima che ci saranno pochissimi utenti bannati \\
Frequenza & Relazione &  \num{600} & Stimo almeno un evento periodico per ogni ente \\
Errore & Entità &  \num{10} & Si stimano circa una decina di errori possibili\\
Checkin completo & Relazione &  \num{2000000} & Si stimano 20 viaggi per abbonamento\\
Vendita & Relazione &  \num{1000000} & Si stima che ogni utente attivo compri due servizi ogni giorno in media\\
Checkin fallito & Relazione &  \num{1000} & Si stimano pochi errori durante la fase di check-in \\
Trasporto pubblico & Entità & * & Sono API esterne, ignoro il loro volume

\end{longtblr}
\endgroup

\subsection{Operazioni richieste e la loro frequenza}
In questa fase esponiamo le tabelle delle operazioni utilizzate per costituire una stima delle principali operazioni richieste dal sito. La stima è su base \ul{settimanale}.



\begingroup % localize the following settings  
\setlength{\arrayrulewidth}{0.5mm}
\renewcommand{\arraystretch}{1.5}

% I = Interattiva
% B = Batch

\begin{longtblr}
[
  caption = {Operazioni richieste da tutti gli User},
  label = {tab:Operazioni richieste da tutti gli User},
]{
  colspec = {|X[1]X[8,l]X[3]X[1]X[8,l]|},
  rowhead = 1,
  hlines,
  row{even} = {lightgray},
  row{1} = {ColdPurple},
} 
Cod & Nome Operazione & Freq & Tipo & Descrizione\\
u.1 & Registrarsi alla piattaforma & \num{2100} & I & Registrazione al sito\\ 
u.2 & Accedere tramite login & \num{7000} & I & Login dell'account \\ 
u.3 & Modificare i dati del proprio account & \num{2100} & I & Login dell'account 
\end{longtblr}



\begin{longtblr}
  [
    caption = {Operazioni richieste amministratore},
    label = {tab:Operazioni richieste amministratore},
  ]{
    colspec = {|X[1]X[8,l]X[3]X[1]X[8,l]|},
    rowhead = 1,
    hlines,
    row{even} = {lightgray},
    row{1} = {CornflowerBlue},
  } 
  Cod & Nome Operazione & Freq & Tipo & Descrizione\\
  a.1 & Consultare la lista degli enti & \num{350} & I & Può consultare la lista degli enti registrati \\
  a.2 & Bannare gli altri utenti & \num{5} & I & L'admin può interdire l'accesso alla piattaforma \\ 
  a.3 & Reset delle recensioni & \num{5} & I & Può cancellare le recensioni \\
  a.4 & Consultazione statistiche & \num{400} & B & Può consultare statistiche globali della piattaforma \\
  \end{longtblr}
  

  \begin{longtblr}
    [
      caption = {Operazioni richieste da Fornitore},
      label = {tab:Operazioni richieste da Fornitore},
    ]{
      colspec = {|X[1]X[8,l]X[3]X[1]X[8,l]|},
      rowhead = 1,
      hlines,
      row{even} = {lightgray},
      row{1} = {LightCoral},
    } 
    Cod & Nome Operazione & Freq & Tipo & Descrizione\\
    f.1 & Creare enti & \num{50} & I & Il fornitore crea enti, all'apertura della piattaforma ci può essere un picco di operazioni al giorno ma poi calerà drasticamente \\ 
    f.2 & Associarsi a un ente  & \num{50} & I & Il deve associarsi al suo ente, con un picco iniziale come la creazione enti \\ 
    f.3 & Creare servizi & \num{50} & I & Il fornitore va a creare servizi \\
    f.4 & Creare eventi occasionali & \num{20} & I & Il fornitore crea eventi \\ 
    f.5 & Creare eventi periodici & \num{5} & I & Il fornitore crea eventi \\ 
    f.6 & Consultare statistiche riguardo il proprio ente  & \num{4000} & B & Il fornitore va a consultare una tabella contenente i dati\\ 
    
    \end{longtblr}


\begin{longtblr}
[
  caption = {Operazioni richieste dai Clienti},
  label = {tab:Operazioni richieste da cliente},
]{
  colspec = {|X[1]X[8,l]X[3]X[1]X[8,l]|},
  rowhead = 1,
  hlines,
  row{even} = {lightgray},
  row{1} = {ColdPurple},
} 
Cod & Nome Operazione & Freq & Tipo & Descrizione\\
c.1 & Richiedere una CityCard & \num{2000} & I & Il cliente può richiedere una CityCard \\ 
c.2 & Sottoscrivere un abbonamento & \num{2000} & I & Il cliente sottoscrive un abbonamento per utilizzare la piattaforma \\ 
c.3 & Aggiungere una carta di credito & \num{2000} & I & Il cliente memorizza una carta di credito nella piattaforma \\ 
c.4 & Rendere una carta di credito predefinita & \num{2000} & I & Il cliente rende predefinita la carta che utilizzera per gli acquisti \\ 
c.5 & Acquistare un servizio & \num{10000} & I & Il cliente può acquistare un servizio \\ 
c.6 & Prenotare un evento & \num{600} & I & Chiede all'applicativo una disponibilità di eventi \\
c.7 & Effettuare un check-in & \num{40000} & I & Chiede all'applicativo una disponibilità di eventi \\ 
c.8 & Consultare la lista degli acquisti fatti & \num{20000} & B & Consulta una lista \\ 
c.9 & Lasciare una recensione riguardo un servizio acquistato & \num{1500} & I & Il cliente rilascia una votazione \\ 
c.9 & Visualizzare lista servizi & \num{30000} & I & Il cliente visualizza la lista dei servizi disponibili \\ 
c.9 & Visualizzare lista eventi & \num{2000} & I & Il cliente visualizza la lista degli eventi disponibili \\ 

\end{longtblr}



\endgroup

\subsection{Schemi di navigazione}
Dopo aver determinato il volume dei dati ed aver associato a ciascuna operazione principale richiesta la frequenza di esecuzione procediamo ad esaminare gli schemi di navigazione per le principali operazioni richieste.

\subsubsection*{u.1 Registrazione di un nuovo utente}
L'utente si registra inserendo i dati minimi richiesti, in seguito potrà aggiungere altri campi modificando il proprio profilo.
\begin{longtblr}
[
  caption = {Registrazione di un nuovo utente},
]{
  colspec = {|X[3]X[1]X[2]X[4]|},
  rowhead = 1,
  hlines,
  row{even} = {PaleTurquoise},
  row{1} = {SkyBlue},
} 
Concetto & Costrutto & Accessi & Tipo\\
Users & E & 1 & Scrittura \\
\SetCell[c=4]{l, white} {
  Totale: 1S \textrightarrow 300/giorno\\
  Costo totale: 300 x (1x2) = 600/giorno
  }

\end{longtblr}


\subsubsection*{u2. Login di un utente}
Per consentire il login si controlla che la combinazione username/password sia corretta e che l'utente non sia bannato. 
\begin{longtblr}
[
  caption = {Login di un utente},
]{
  colspec = {|X[3]X[1]X[2]X[4]|},
  rowhead = 1,
  hlines,
  row{even} = {PaleTurquoise},
  row{1} = {SkyBlue},
} 
Concetto & Costrutto & Accessi & Tipo\\
Users & E & 1 & L\\ 
\SetCell[c=4]{l, white} {
  Totale: 1L \textrightarrow 1500/giorno\\
  Costo totale: 1500 x (1) = 1500/giorno
  }

\end{longtblr}

\subsubsection*{u3. Modificare i dati del proprio account}
L'utente modifica i propri dati oppure inserisce quelli mancanti.
\begin{longtblr}
  [
    caption = {Modificare i dati del proprio account},
  ]{
    colspec = {|X[3]X[1]X[2]X[4]|},
    rowhead = 1,
    hlines,
    row{even} = {PaleTurquoise},
    row{1} = {SkyBlue},
  } 
  Concetto & Costrutto & Accessi & Tipo\\
  Users & E & 1 & S\\ 
  \SetCell[c=4]{l, white} {
  Totale: 1S \textrightarrow 300/giorno\\
  Costo totale: 300 x (1x2) = 600/giorno
  }
  \end{longtblr}

%%%%%%%%%%%%%%%%%%%%
%%%%%% ADMIN %%%%%%%
%%%%%%%%%%%%%%%%%%%%
\subsubsection*{a1. Consultare la lista degli utenti}
L'amministratore può visualizzare gli utenti registrati e bannarli o riattivarli all'occorrenza.
\begin{longtblr}
  [
    caption = {Consultare la lista degli enti},
  ]{
    colspec = {|X[3]X[1]X[2]X[4]|},
    rowhead = 1,
    hlines,
    row{even} = {lightgray},
    row{1} = {LightCoral},
  } 
  Concetto & Costrutto & Accessi & Tipo\\
  User & E & 1 & L\\  
  ban & R & 1 & L\\ 
  \SetCell[c=4]{l, white} {
    Totale: 2L \textrightarrow 50/giorno\\
    Costo totale: 50 x (2) = 100/giorno
    }

  \end{longtblr}


  \subsubsection*{a2. Bannare gli altri utenti}
  Un amministratore banna un utente impedendone i futuri login.
  \begin{longtblr}
    [
      caption = {Bannare gli altri utenti},
    ]{
      colspec = {|X[3]X[1]X[2]X[4]|},
      rowhead = 1,
      hlines,
      row{even} = {lightgray},
      row{1} = {LightCoral},
    } 
    Concetto & Costrutto & Accessi & Tipo\\
    User & E & 1 & L\\
    ban & R & 1 & S \\ 

    \SetCell[c=4]{l, white} {
      Totale: 1L + 1S \textrightarrow 5/giorno\\
      Costo totale: 5 x (1 + 1 * 2) = 15/giorno
      }
  \end{longtblr}


\subsubsection*{a3. Reset delle recensioni}
In caso di necessità è possibile cancellare tutte le recensioni di un ente.
\begin{longtblr}
[
caption = {Reset delle recensioni},
]{
colspec = {|X[3]X[1]X[2]X[4]|},
rowhead = 1,
hlines,
row{even} = {lightgray},
row{1} = {LightCoral},
} 
Concetto & Costrutto & Accessi & Tipo\\
Ente & E & 1 & L \\
vendita & R & 1 & L \\
Servizio & E & 8 & L\\ 
Servizio & E & 8 & S\\ 
Recensione & R & 1600 (200 * 8) & L \\
Recensione & R & 1600 & S \\

\SetCell[c=4]{l, white} {
    Totale: 1610L + 1608S \textrightarrow 1/giorno\\
    Costo totale: 1 x (1610L + 1608S * 2) = 4,826/giorno
    }
\end{longtblr}



\subsubsection*{a4. Consultare le statistiche}
Verranno eseguite le query necessarie per ottenere le seguenti statistiche:\\
\begin{itemize}
  \item numero check-in
  \item numero check-in falliti
  \item numero CityCard attive
  \item numero eventi attivi
  \item numero servizi attivi
\end{itemize}
\begin{longtblr}
[
caption = {Consultare statistiche},
]{
colspec = {|X[3]X[1]X[2]X[4]|},
rowhead = 1,
hlines,
row{even} = {lightgray},
row{1} = {LightCoral},
} 
Concetto & Costrutto & Accessi & Tipo\\
checkin{\_}completato & R & 1 & L \\
checkin{\_}falliti & R & 2 & L \\
possesso{\_}CityCard & R & 1 & L \\
vendita & R & 1 & L \\
organizzazione & R & 1 & L \\

\SetCell[c=4]{l, white} {
    Totale: 5L \textrightarrow 50/giorno\\
    Costo totale: 50 x (5) = 250/giorno
    }
\end{longtblr}


%%%%%%%%%%%%%%%%%%%%%%%%
%%%%%% FORNITORE %%%%%%%
%%%%%%%%%%%%%%%%%%%%%%%%

\subsubsection*{f1. Creare enti}
Ogni fornitore può creare uno o più enti.
\begin{longtblr}
[
caption = {Creare enti},
]{
colspec = {|X[3]X[1]X[2]X[4]|},
rowhead = 1,
hlines,
row{even} = {lightgray},
row{1} = {ColdPurple},
} 
crea{\_}ente & R & 1 & S \\
\SetCell[c=4]{l, white} {
Totale: 1S \textrightarrow 8/giorno\\
Costo totale: 8 x (1 x 2) = 16/giorno
}
\end{longtblr}


\subsubsection*{f2. Associarsi a un ente}
Il fornitore si può associare a un solo ente.
\begin{longtblr}
[
caption = {Associarsi a un ente},
]{
colspec = {|X[3]X[1]X[2]X[4]|},
rowhead = 1,
hlines,
row{even} = {lightgray},
row{1} = {ColdPurple},
} 
Concetto & Costrutto & Accessi & Tipo\\
lavoro & R & 1 & S \\ 
\SetCell[c=4]{l, white} {
    Totale: 1S \textrightarrow 3/giorno\\
    Costo totale: 3 x (1 x 2) = 6/giorno
    }
\end{longtblr}

\subsubsection*{f3. Creare servizi}
Ogni fornitore associato ad un ente può creare servizi.
\begin{longtblr}
[
caption = {Creare servizi},
]{
colspec = {|X[3]X[1]X[2]X[4]|},
rowhead = 1,
hlines,
row{even} = {lightgray},
row{1} = {ColdPurple},
} 
Concetto & Costrutto & Accessi & Tipo\\
vendita & R & 1 & S \\
lavoro & R & 1 & L \\
\SetCell[c=4]{l, white} {
    Totale: 1L + 1S \textrightarrow 10/giorno\\
    Costo totale: 10 x (1 + 1 x 2) = 30/giorno
    }
\end{longtblr}


\subsubsection*{f4. Creare eventi occasionali}
\includegraphics[width=0.95\columnwidth]{f4creaEventoOccasionale.png}\\
I fornitori associati a un ente possono creare eventi occasionali.
\begin{longtblr}
[
caption = {Creare eventi occasionali},
]{
colspec = {|X[3]X[1]X[2]X[4]|},
rowhead = 1,
hlines,
row{even} = {lightgray},
row{1} = {ColdPurple},
} 
Concetto & Costrutto & Accessi & Tipo\\
lavoro & R & 1 & L \\
organizza & R & 1 & S \\
Evento & E & 1 & S \\
Occasionale & E & 1 & S \\
\SetCell[c=4]{l, white} {
    Totale: 1L + 3S \textrightarrow 4/giorno\\
    Costo totale: 4 x (1 + 3 x 2) = 28/giorno
    }
\end{longtblr}




\subsubsection*{f5. Creare eventi periodici}
\includegraphics[width=0.95\columnwidth]{f5creaEventoPeriodico.png}\\
Per creare un evento periodico si dovrà ottenere l'id dell'ente al quale è associato il fornitore, poi andare a creare un record per l'evento e un altro record per il periodo. \\
\begin{longtblr}
[
caption = {Creare eventi periodici},
]{
colspec = {|X[3]X[1]X[2]X[4]|},
rowhead = 1,
hlines,
row{even} = {lightgray},
row{1} = {ColdPurple},
} 
Concetto & Costrutto & Accessi & Tipo\\
lavoro & R & 1 & L \\
organizza & R & 1 & S \\
Evento & E & 1 & S \\
Periodico & E & 1 & L \\
frequenza & R & 1 & S \\
Calendario & E & 1 & S \\
\SetCell[c=4]{l, white} {
    Totale: 2L + 4S \textrightarrow 1/giorno\\
    Costo totale: 1 x (2 + 4 x 2) = 10/giorno
    }
\end{longtblr}



\subsubsection*{f6. Consultare statistiche riguardo il proprio ente}
Oltre alla query per ottenere l'id dell'ente del fornitore verranno eseguite quelle necessarie per ottenere le seguenti statistiche:\\
\begin{itemize}
  \item saldo dell'ente associato al fornitore
  \item numero eventi attivi
  \item numero servizi attivi
\end{itemize}

\begin{longtblr}
[
caption = {Consultare statistiche riguardo il proprio ente},
]{
colspec = {|X[3]X[1]X[2]X[4]|},
rowhead = 1,
hlines,
row{even} = {lightgray},
row{1} = {ColdPurple},
} 
Concetto & Costrutto & Accessi & Tipo\\
Ente & E & 1 & L \\
Eventi & E & 1 & L\\ 
Servizi & E & 1 & L\\ 
\SetCell[c=4]{l, white} {
    Totale: 2L \textrightarrow 800/giorno\\
    Costo totale: 800 x (3) = 2400/giorno
    }
\end{longtblr}

%%%%%%%%%%%%%%%%%%%%%%
%%%%%% CLIENTE %%%%%%%
%%%%%%%%%%%%%%%%%%%%%%
\subsubsection*{c1. Richiedere una CityCard}
Il cliente richiede una nuova CityCard.
\begin{longtblr}
[
caption = {Richiedere una CityCard},
]{
colspec = {|X[3]X[1]X[2]X[4]|},
rowhead = 1,
hlines,
row{even} = {lightgray},
row{1} = {MediumSeaGreen},
} 
Concetto & Costrutto & Accessi & Tipo \\
Possesso CityCard & R & 1 & S \\
CityCard & E & 1 & S \\
\SetCell[c=4]{l, white} {
    Totale: 2S \textrightarrow 2000/giorno\\
    Costo totale: 2000 x (2 x 2) = 8000/giorno
    }
\end{longtblr}

\subsubsection*{c2. Sottoscrivere un abbonamento}
Prima di sottoscrivere un abbonamento vengono cercate una CityCard valida e una carta di credito predefinita. \\
\includegraphics[width=0.95\columnwidth]{c2sottoscrizioneAbbonamento.png}\\

\begin{longtblr}
[
caption = {Sottoscrivere un abbonamento},
]{
colspec = {|X[3]X[1]X[2]X[4]|},
rowhead = 1,
hlines,
row{even} = {lightgray},
row{1} = {MediumSeaGreen},
} 
Concetto & Costrutto & Accessi & Tipo \\
Cliente & E & 1 & L\\ 
possesso{\_}carta{\_}credito & R & 1 & L \\
Carta{\_}Credito & E & 1 & L \\
possesso{\_}citycard & R & 1 & L \\
CityCard & E & 1 & L \\
sottoscrizione & R & 1 & S \\
\SetCell[c=4]{l, white} {
    Totale: 5L + 1S \textrightarrow 2000/giorno\\
    Costo totale: 2000 x (5 + 1 x 2) = 16000/giorno
    }
\end{longtblr}


\subsubsection*{c3. Aggiungere una carta di credito}
Un cliente può salvare le proprie carte di credito.
\begin{longtblr}
[
caption = {Aggiungere una carta di credito},
]{
colspec = {|X[3]X[1]X[2]X[4]|},
rowhead = 1,
hlines,
row{even} = {lightgray},
row{1} = {MediumSeaGreen},
} 
Concetto & Costrutto & Accessi & Tipo \\
Carta credito & R & S \\
possesso{\_}carta{\_}credito & R & 1 & S \\
\SetCell[c=4]{l, white} {
    Totale: 2S \textrightarrow 2000/giorno\\
    Costo totale: 2000 x (2 x 2) = 8000/giorno
    }
\end{longtblr}


\subsubsection*{c4. Rendere una carta di credito predefinita}
Rendo predefinita una carta di credito di un utente e non predefinite tutte le altre.\\
(dato che in media ho 1,5 carte per cliente, approssimo a 2)
\begin{longtblr}
[
caption = {Aggiungere una carta di credito},
]{
colspec = {|X[3]X[1]X[2]X[4]|},
rowhead = 1,
hlines,
row{even} = {lightgray},
row{1} = {MediumSeaGreen},
} 
Concetto & Costrutto & Accessi & Tipo \\
possesso{\_}carta{\_}credito & R & 2 & L \\
Carta credito & R & 2 & L \\
Carta credito & R & 2 & S \\
\SetCell[c=4]{l, white} {
    Totale: 4L + 2S \textrightarrow 2000/giorno\\
    Costo totale: 2000 x (4 + 2 x 2) = 16000/giorno
    }
\end{longtblr}


\subsubsection*{c5. Acquistare un servizio}
Viene controllata la CityCard dell'utente, tramite essa vengono recuperati i dati della sottoscrizione, poi quelli dello sconto, con questi dati viene calcolato il prezzo e infine viene comprato il servizio e aggiornato il saldo dell'ente.
\includegraphics[width=0.95\columnwidth]{c5acquistaServizio.png}\\

\begin{longtblr}
[
caption = {Acquistare un servizio},
]{
colspec = {|X[3]X[1]X[2]X[4]|},
rowhead = 1,
hlines,
row{even} = {lightgray},
row{1} = {MediumSeaGreen},
} 
Concetto & Costrutto & Accessi & Tipo \\
possesso{\_}citycard & R & 1 & L \\
CityCard & E & 1 & L\\ 
sottoscrizione & R & 1 & L \\
Abbonamento & E & 1 & L\\ 
riduzione & R & 1 & L \\
Sconto & E & 1 & L\\ 
possesso{\_}carta{\_}credito & R & 1 & L \\
Carta{\_}credito & E & 1 & L \\
transazione & R & 1 & S\\ 
Servizio & E & 1 & L\\ 
vendita & R & 1 & L\\ 
Ente & E & 1 & S\\ 
\SetCell[c=4]{l, white} {
    Totale: 10L + 2S \textrightarrow 3000/giorno\\
    Costo totale: 3000 x (10 + 2 x 2) = 42000/giorno
    }
\end{longtblr}

\subsubsection*{c6. Prenotare un evento}
Gli utenti possono prenotare gratuitamente degli eventi.
\begin{longtblr}
[
caption = {Prenotare un evento},
]{
colspec = {|X[3]X[1]X[2]X[4]|},
rowhead = 1,
hlines,
row{even} = {lightgray},
row{1} = {MediumSeaGreen},
} 
Concetto & Costrutto & Accessi & Tipo \\
partecipazione{\_}persona & R & 1 & S \\
\SetCell[c=4]{l, white} {
    Totale: 1S \textrightarrow 500/giorno\\
    Costo totale: 500 x (1 x 2) = 1000/giorno
    }
\end{longtblr}


\subsubsection*{c7. Effettuare un check-in}
Per poter effettuare un check-in devo confermare che la CityCard del cliente sia valida.
\begin{longtblr}
[
caption = {Effettuare un check-in},
]{
colspec = {|X[3]X[1]X[2]X[4]|},
rowhead = 1,
hlines,
row{even} = {lightgray},
row{1} = {MediumSeaGreen},
} 
Concetto & Costrutto & Accessi & Tipo \\
possesso{\_}citycard & R & 1 & L \\
CityCard & E & 1 & L\\ 
checkin(completato/fallito) & R & 1 & S \\
\SetCell[c=4]{l, white} {
    Totale: 2L + 1S \textrightarrow 5000/giorno\\
    Costo totale: 5000 x (2 + 1 x 2) = 20000/giorno
    }
\end{longtblr}

\subsubsection*{c8. Consultare la lista degli acquisti fatti}
Un utente può visualizzare la lista di tutti gli acquisti fatti.
\begin{longtblr}
[
caption = {Consultare la lista degli acquisti fatti},
]{
colspec = {|X[3]X[1]X[2]X[4]|},
rowhead = 1,
hlines,
row{even} = {lightgray},
row{1} = {MediumSeaGreen},
} 
Concetto & Costrutto & Accessi & Tipo \\
possesso{\_}citycard & R & 1 & L \\
CityCard & E & 1 & L\\ 
transazione & R & 1 & L\\ 
\SetCell[c=4]{l, white} {
    Totale: 3L \textrightarrow 3000/giorno\\
    Costo totale: 3000 x (3) = 9000/giorno
    }
\end{longtblr}


\subsubsection*{c9. Lasciare una recensione riguardo un servizio acquistato}
Questa funzionalità verrà approfondita nella prossima sezione dedicata al calcolo delle ridondanze.
% \begin{longtblr}
% [
% caption = {Lasciare una recensione riguardo un servizio acquistato},
% ]{
% colspec = {|X[3]X[1]X[2]X[4]|},
% rowhead = 1,
% hlines,
% row{even} = {lightgray},
% row{1} = {MediumSeaGreen},
% } 
% Concetto & Costrutto & Accessi & Tipo \\
% servizi acquistati & E & 1 & L\\ 
% Cliente & E & 1 & L\\ 
% Recensione & R & 1 & S \\
% Servizio & E & 1 & L \\

% \SetCell[c=4]{l, white} {
%     Totale: 2L + 1S \textrightarrow 1500/giorno\\
%     Costo totale: 3 x (1500) = 4500/giorno
%     }
% \end{longtblr}



\subsubsection*{c10. Visualizzare lista servizi}
Anche questa funzionalità verrà approfondita nella prossima sezione dedicata al calcolo delle ridondanze.

% \includegraphics[width=0.95\columnwidth]{c10getServizi.png}\\
% Nella tabella dei servizi disponibili si dovrà mettere anche l'ente fornitore, il prezzo scontato in base all'abbonamento sottoscritto dal cliente.
% \begin{longtblr}
% [
% caption = {Visualizzare lista servizi},
% ]{
% colspec = {|X[3]X[1]X[2]X[4]|},
% rowhead = 1,
% hlines,
% row{even} = {lightgray},
% row{1} = {MediumSeaGreen},
% } 
% Concetto & Costrutto & Accessi & Tipo \\
% possesso{\_}citycard & R & 1 & L \\
% CityCard & E & 1 & L\\ 
% sottoscrizione & R & 1 & L \\
% Abbonamento & E & 1 & L\\ 
% riduzione & R & 1 & L \\
% Sconto & E & 1 & L\\ 
% Servizio & E & 1 & L\\ 
% vendita & R & 1 & L \\
% Ente & E & 1 & L \\

% \SetCell[c=4]{l, white} {
%     Totale: 9L \textrightarrow 7500/giorno\\
%     Costo totale: 3 x (7500) = 22500/giorno
%     }
% \end{longtblr}


\subsubsection*{c11. Visualizzare lista eventi}
\includegraphics[width=0.95\columnwidth]{c11getEventi.png}\\
L'utente visualizza la lista di tutti gli eventi disponibili, sia periodici che occasionali.
\begin{longtblr}
[
caption = {Visualizzare lista eventi},
]{
colspec = {|X[3]X[1]X[2]X[4]|},
rowhead = 1,
hlines,
row{even} = {lightgray},
row{1} = {MediumSeaGreen},
} 
Concetto & Costrutto & Accessi & Tipo \\
Ente & E & 1 & L \\
organizzazione & R & 1 & L \\
Occasionale & E & 1 & L\\ 
Periodico & E & 1 & L\\ 
frequenza & R & 1 & L \\
Calendario & E & 1 & L\\ 

\SetCell[c=4]{l, white} {
    Totale: 6L \textrightarrow 1500/giorno\\
    Costo totale: 1500 x (6) = 9000/giorno
    }
\end{longtblr}


\subsection{Analisi delle ridondanze}
% TODO sostituire partecipanti con recensioni

In questa fase ci occuperemo dell'analisi delle ridondanze cioè quella informazioni che possono essere ricavate altrove.
Una ridondanza quindi corrisponde ad un dato che può essere derivato, cioè ottenuto attraverso una serie di operazioni, da altri dati.
Se si mantiene la ridondanza si semplificano alcune interrogazioni ma si occupa maggior spazio.
All'interno del nostro schema possiamo fare alcune considerazioni.

\begin{itemize}
    \item L'attributo numero{\_}partecipanti è derivabile da una operazione di conteggio delle istanze di clienti che hanno prenotato un evento.
    Dobbiamo tenere conto delle frequenze di esecuzione.
    \begin{itemize}
        \item operazione 1: si tiene il conteggio ogni volta che un cliente prenota un evento
        \item operazione 2: vengono visualizzati tutti i dati di un evento
    \end{itemize}
\end{itemize}

\subsection{Raffinamento dello schema} 
\subsubsection{Nomenclatura tabelle}
Si è deciso di standardizzare i nomi di tutte le tabelle scegliendo sostantivi plurali.

\subsubsection{Eliminazione delle gerarchie di generalizzazione}
Nel processo di definizione del nostro schema concettuale, abbiamo identificato alcune gerarchie di generalizzazione che richiedevano un'analisi approfondita per determinare la loro utilità e struttura all'interno del modello. Dopo un'attenta valutazione, abbiamo deciso di semplificare queste gerarchie, adottando una strategia di riduzione che ha portato all'eliminazione delle generalizzazioni e alla fusione degli attributi dei sottotipi nelle entità principali. Di seguito, illustriamo le specifiche modifiche effettuate:

\begin{itemize} 
\item 
Generalizzazione dell’entità \textit{User}: Inizialmente, l’entità \textit{User} era rappresentata come una generalizzazione totale ed esclusiva con tre sottotipi distinti: \textit{Cliente}, \textit{Admin} e \textit{Fornitore}. Tuttavia, durante la fase di progettazione, abbiamo riscontrato che tutti i sottotipi condividevano gli stessi attributi fondamentali, rendendo superflua la distinzione tra essi. Pertanto, abbiamo deciso di far collassare la gerarchia verso l’alto, integrando gli attributi direttamente nell’entità \textit{User}. Questo approccio ha semplificato la struttura del modello, migliorando la chiarezza e l'efficienza nella gestione dei dati senza perdere informazioni rilevanti.
\begin{center}
    \includegraphics[]{UserLogico.png}
\end{center}

\item 
Generalizzazione dell’entità \textit{Eventi}: 
Analogamente, l’entità \textit{Eventi} era inizialmente strutturata come una generalizzazione totale ed esclusiva, comprendente i sottotipi \textit{Periodico} e \textit{Occasionale}, utilizzati per definire la natura dell'evento. Tuttavia, analizzando l’architettura del sistema, abbiamo ritenuto più efficiente eliminare la gerarchia e incorporare i sottotipi come attributi dell’entità principale \textit{Eventi}. Questa scelta ha permesso di mantenere la flessibilità nella classificazione degli eventi, riducendo al contempo la complessità del modello e facilitando la gestione delle informazioni relative agli eventi.
\begin{center}
    \includegraphics[]{EventiLogico.png}
\end{center}
\end{itemize}

\subsubsection{Chiavi esterne}
Vengono utilizzate, per semplicità, chiavi esterne con lo stesso nome della chiave primaria a cui fanno riferimento.






\subsection{Traduzione delle entità}
\begin{itemize}
    \item 
    \textbf{Carta{\_}credito}(\underline{num{\_}carta{\_}credito}, cognome{\_}associato, nome{\_}associato,
    mese{\_}scadenza, anno{\_}scadenza, data{\_}registrazione{\_}carta, id{\_}user,
    predefinita)\\
    FK: id{\_}user \textrightarrow USERS.id{\_}user

    \item 
    \textbf{Checks}(
    \underline{id{\_}check}, 
    orario{\_}convalida
    id{\_}city{\_}card
    id{\_}mezzo
    id{\_}stato)\\
    FK: id{\_}mezzo \textrightarrow MEZZI.id{\_}mezzo\\
    FK: id{\_}stato \textrightarrow STATI{\_}CHECK.id{\_}stato\\
    FK: id{\_}city{\_}card \textrightarrow CITY{\_}CARD.id{\_}city{\_}card\\

    \item 
    \textbf{CityCard}
    (\underline{id{\_}city{\_}card},
    id{\_}utente,
    data{\_}emissione,
    data scadenza)

    \item 
    \textbf{Collaborazioni}
    (
    \underline{id{\_}collaborazione}
    inizio{\_}collaborazione
    fine{\_}collaborazione
    id{\_}user
    id{\_}ente)\\
    FK: id{\_}user \textrightarrow USER.id{\_}user\\
    FK: id{\_}ente \textrightarrow ENTI.id{\_}ente

    \item
    \textbf{Enti}
    (
    \underline{id{\_}ente}
    descrizione{\_}ente,
    saldo,
    indirizzo,
    numero{\_}telefono,
    nome,
    id{\_}user)\\
    FK: id{\_}user \textrightarrow USERS id{\_}user
    
    \item 
    \textbf{Eventi}
    (\underline{id{\_}evento},
    nome{\_}evento,
    num{\_}partecipanti,
    id{\_}periodo,
    inizio{\_}validità,
    fine{\_}validità,
    id{\_}ente)\\
    FK: id{\_}periodo \textrightarrow PERIODI.id{\_}periodo,\\
    FK: id{\_}ente \textrightarrow ENTI.id{\_}ente
    
    \item 
    \textbf{Listino abbonamenti}
    (
    \underline{id{\_}listino{\_}abbonamento},
    descrizione{\_}abbonamento,
    prezzo{\_}abbonamento,
    durata{\_}abbonamento,
    data{\_}disattivazione,
    id{\_}sconto)\\
    FK: id{\_}sconto \textrightarrow SCONTI.id{\_}sconto
    
    \item 
    \textbf{Mezzi}
    (
    \underline{id{\_}mezzo},
    desc{\_}mezzo
    partenza
    destinazione
    )

    \item 
    \textbf{Partecipazioni}
    (
    \underline{id{\_}partecipazione},
    data{\_}registrazione,
    id{\_}evento,
    id{\_}city{\_}card)\\
    FK: id{\_}city{\_}card \textrightarrow CITY{\_}CARD.id{\_}city{\_}card,\\
    FK: id{\_}evento \textrightarrow EVENTI.id{\_}evento

    \item   
    \textbf{Periodi}
    (\underline{id{\_}periodo},
    lunedi,
    martedi,
    mercoledi,
    giovedi,
    venerdi,
    sabato,
    domenica)

    \item 
    \textbf{Recensioni}
    (\underline{id{\_}recensione},
    data{\_}inserimento,
    votazione,
    id{\_}servizio,
    id{\_}user)\\
    FK: id{\_}servizio \textrightarrow SERVIZI.id{\_}servizio,\\
    FK: id{\_}user \textrightarrow USERS.id{\_}user

    \item 
    \textbf{Ruoli}
    (\underline{id{\_}ruolo},
    descrizione{\_}ruolo
    )

    \item 
    \textbf{Sconti}
    (
    \underline{id{\_}sconto},
    percentuale{\_}sconto)

    \item 
    \textbf{Servizi}
    (
    \underline{id{\_}servizio},
    prezzo
    data{\_}inserimento
    data{\_}termine
    descrizione{\_}servizio
    indirizzo{\_}servizio
    media{\_}recensioni
    id{\_}ente)\\
    FK: id{\_}ente \textrightarrow ENTI.id{\_}ente
    
    \item 
    \textbf{Servizi acquistati}
    (
    \underline{id{\_}acquisto},
    data{\_}acquisto,
    prezzo{\_}pagato,
    media{\_}recensioni,
    num{\_}carta{\_}credito,
    id{\_}city{\_}card,
    id{\_}servizio)\\
    FK: id{\_}servizio \textrightarrow SERVIZI.id{\_}servizio\\
    FK: id{\_}city{\_}card \textrightarrow CITY{\_}CARD.id{\_}city{\_}card\\
    FK: num{\_}carta{\_}credito \textrightarrow CARTE{\_}CREDITO.num{\_}carta{\_}credito
    
    \item 
    \textbf{Sottoscrizioni abbonamento},
    (
    \underline{id{\_}sottoscrizione{\_}abbonamento},
    scadenza{\_}sottoscrizione,
    data{\_}sottoscrizione,
    id{\_}citycard,
    num{\_}carta{\_}credito)\\
    FK: id{\_}listino{\_}abbonamento \textrightarrow LISTINO{\_}ABBONAMENTI.id{\_}listino{\_}abbonamento\\
    FK: id{\_}city{\_}card \textrightarrow CITY{\_}CARD.id{\_}city{\_}card\\
    FK: num{\_}carta{\_}credito \textrightarrow CARTE{\_}CREDITO.num{\_}carta{\_}credito
    
    \item 
    \textbf{Stati check}
    (
    \underline{id{\_}stato},
    desc{\_}stato
    )
    \item 
    \textbf{Users}
    (\underline{id{\_}user},
    username,
    password,
    email,
    CF,
    nome,
    cognome,
    telefono,
    indirizzo,
    creazione,
    bannato,
    citta,
    CAP,
    nazione,
    data{\_}creazione,
    id{\_}ruolo)\\
    FK: id{\_}ruolo \textrightarrow RUOLI.id{\_}ruolo
    
\end{itemize}




\subsection{Traduzione delle associazioni}
\begin{itemize}
    \item 
    \textbf{Ban} (Lega ADMIN e USER) \textrightarrow \thinspace 
    diventa attributo "bannato" nell'entità USERS.

    \item 
    \textbf{Creazione ente} (lega FORNITORE e ENTE) \textrightarrow \thinspace 
    diventa chiave esterna nell'entità ENTI.

    \item 
    \textbf{Lavoro} (lega FORNITORE e ENTE) \textrightarrow \thinspace 
    diventa la nuova entità COLLABORAZIONI.

    \item 
    \textbf{Organizzazione} (lega ENTE e EVENTO) \textrightarrow \thinspace 
    diventa chiave esterna nell'entità EVENTI.

    \item 
    \textbf{Partecipazione} (lega CLIENTE e EVENTO) \textrightarrow \thinspace
    diventa la nuova entità PARTECIPAZIONI.

    \item 
    \textbf{Possiede carta di credito} (lega  CLIENTE e CARTA DI CREDITO) \textrightarrow \thinspace 
    diventa chiave esterna nell'entità CARTE CREDITO.

    \item 
    \textbf{Possiede CityCard} (lega CLIENTE e CITYCARD) \textrightarrow \thinspace 
    diventa chiave esterna nell'entità CITY CARD.

    \item 
    \textbf{Transazione} (lega CARTA DI CREDITO e CITYCARD e SERVIZIO) \textrightarrow \thinspace 
    diventa la nuova entità SERVIZI ACQUISTATI.

    \item 
    \textbf{Sottoscrive} (lega ABBONAMENTO e CITYCARD e CARTA DI CREDITO)  \textrightarrow \thinspace 
    diventa la nuova entità SOTTOSCRIZIONI ABBONAMENTO.

    \item 
    \textbf{Riduzione} (lega ABBONAMENTO e SCONTO) \textrightarrow \thinspace 
    diventa chiave esterna nell'entità ABBONAMENTI.
    
    \item 
    \textbf{Checkin completato} e \textbf{Checkin fallito} (lega CITY CARD e TRASPORTO PUBBLICO)  \textrightarrow \thinspace 
    diventano la nuova entità CHECKS.
    
    \item 
    \textbf{Recensione} (lega CLIENTE e SERVIZIO)  \textrightarrow \thinspace 
    diventano la nuova entità RECENSIONI.
    
    \item 
    \textbf{Vendita} (lega ENTE e SERVIZIO)  \textrightarrow \thinspace 
    diventa chiave esterna nell'entità SERVIZI.
    
    \item 
    \textbf{Frequenza} (lega PERIODICO e CALENDARIO)  \textrightarrow \thinspace 
    diventa chiave esterna nell'entità EVENTI.

    
\end{itemize}


\subsection{Schema relazionale finale}
\begin{center}
    \centerline{\includegraphics[width=0.9\paperwidth]{schema_logico.png}}
\end{center}

\subsection{Traduzioni delle operazioni in Query SQL}
\input{doc/3_9_Traduzioni_delle_operazioni_in_query_SQL}



\section{Progettazione dell'applicazione}
\subsection{Descrizione dell'architettura dell'applicazione}
L'interfaccia utente del nostro elaborato è un prototipo di ciò che sarà effettivamente la piattaforma.
L'applicazione è stata realizzata in linguaggio Javascript avvalendoci dei fogli di stile CSS e di Bootstrap per quanto concerne l'aspetto grafico e linguaggio Mysql per interrogare il database.
Ci si deve occupare di gestire tre tipologie di utenti: gli amministratori, i fornitori e i clienti.
Nella schermata iniziale di registrazione sarà possibile scegliere il proprio ruolo all'interno del sito, ottenendo privilegi o accesso a pagine del tutto diverse tra loro.
Passiamo ora ad esminare questo tipo di interfacce e le diverse funzioni da esse offerte.
 

\subsection{Utente Generico}
\includegraphics[width=0.95\columnwidth]{utente_generico_header.png}\\
Nella schermata di tutti gli utenti sarà presente un header che darà il benvenuto, il nome visualizzato sarà quello con il quale è stata fatta la registrazione. Sarà possibile effettuare il logout e accedere alle informazioni del proprio profilo \\
A livello di applicativo durante il login vengono salvate nella sessione le informazioni riguardanti l'utente. Un esempio è proprio il nome che appare nell'header, oppure l'id utente che verrà utilizzato per numerose altre query.\\
\includegraphics[width=0.95\columnwidth]{utente_generico_modifica.png}\\
Cliccando sul tasto "Profilo" ogni utente potrà visualizzare e modificare le informazioni del proprio account.


\subsection{L'amministratore}
\begin{center}
    \fbox{\includegraphics[width=0.95\columnwidth]{amministratore_home.png}}
\end{center}
Una volta effettuato l'accesso l'applicativo offre all'amministratore le seguenti possibilità:
\begin{itemize}
    \item Visualizzare gli enti
    \item Visualizzare gli utenti
    \item Consultazione delle statistiche
\end{itemize}
\begin{center}
    \fbox{\includegraphics[width=0.95\columnwidth]{amministratore_enti.png}}
\end{center}
Il tasto "Visualizza gli enti" permette di poter visionare l'elenco degli enti registrati all'interno della piattaforma con le relative informazioni e al tasto per resettarne le recensioni, sarà presente un tasto aggiorna in cui si avrà accesso alla lista più recente delle aziende partner.
\begin{center}
    \fbox{\includegraphics[width=0.95\columnwidth]{amministratore_utenti.png}}
\end{center}
La funzione sarà analoga anche per la lista degli utenti registrati, sarà inoltre possibile interdire l'accesso al sito tramite il tasto "Banna". Una volta bannato l'utente non potrà più effettuare il login. Se l'utente è bannato il tasto cambierà e potra essere riattivato.
\begin{center}
    \fbox{\includegraphics[width=0.95\columnwidth]{amministratore_statistiche.png}}
\end{center}
L'amministratore ha accesso ad alcune statistiche globali della piattaforma.

\subsection{Il fornitore}
\begin{center}
    \fbox{\includegraphics[width=0.95\columnwidth]{fornitore_home_1.png}}
\end{center}
Una volta effettuato il primo accesso come fornitore l'applicativo offre le seguenti possibilità:
\begin{itemize}
    \item Associarsi ad un ente
    \item Creare un ente
\end{itemize}
\begin{center}
    \fbox{\includegraphics[width=0.95\columnwidth]{fornitore_home_2.png}}
\end{center}
Il tasto "Associa al tuo ente" permette al fornitore di potersi associare ad un ente già registrato all'interno della piattaforma mente il tasto "Crea un ente" permette di creare da zero un ente con il quale associarsi. Le altre funzionalità rimarranno disabilitate fino a che non verrà associato un ente.
\begin{itemize}
    \item Creare un evento
    \item Creare un servizio
    \item Consultazione delle statistiche degli eventi, del saldo e dei servizi
\end{itemize}
\begin{center}
    \fbox{\includegraphics[width=0.95\columnwidth]{fornitore_evento.png}}
\end{center}
\begin{center}
    \fbox{\includegraphics[width=0.95\columnwidth]{fornitore_servizio.png}}
\end{center}
I successivi comandi permettono di creare un evento (sia occasionale che periodico con gli appositi tasti) oppure un servizio.
\begin{center}
    \fbox{\includegraphics[width=0.95\columnwidth]{fornitore_statistiche.png}}
\end{center}
La consultazione delle statistiche saldo e servizi permettono rispettivamente di consultare il saldo accumulato dalla registrazione di eventi e dall'acquisto di servizi.





\subsection{Il cliente}
Se si effettua il primo login in assoluto come cliente gran parte del menù appare disattivato in quanto la maggior parte delle attività si sbloccano in presenza di una CityCard.
Il menù offre le seguenti funzionalità:
\begin{itemize}
    \item Ottieny CityCard
    \item Gestione carte di credito
    \item Sottoscrivi abbonamento
    \item Visualizza eventi
    \item Visualizza servizi
    \item Cronologia acquisti
    \item Simula check in
\end{itemize}
Come detto sopra al momento del primo login le uniche sezioni navigabili sono "Ottieni CityCard" e "Gestione carta di credito" che permettono rispettivamente di ottenere una CityCard e di gestire le carte di credito registrate dal cliente all'interno dell'applicativo.
Una volta ottenuta la tessera sarà disponibile la sezione "Acquista abbonamento" dove sarà possibile scegliere tra tre tipologie di sottoscrizione.  
Una volta scetìlto il tipo di abbonamento anche il resto del menù sarà sbloccato e sarà possibile comprare servizi, prenotarsi ad eventi e visualizzare la cronologia degli acquisti.



\end{document}
